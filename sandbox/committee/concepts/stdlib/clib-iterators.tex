\documentclass[american,twoside]{book}
\usepackage{refbib}
\usepackage{pdfsync}
% Definitions and redefinitions of special commands

\usepackage{babel}      % needed for iso dates
%\usepackage{savesym}		% suppress duplicate macro definitions
\usepackage{fancyhdr}		% more flexible headers and footers
\usepackage{listings}		% code listings
\usepackage{longtable}	% auto-breaking tables
\usepackage{remreset}		% remove counters from reset list
\usepackage{booktabs}		% fancy tables
\usepackage{relsize}		% provide relative font size changes
%\usepackage[htt]{hyphenat}	% hyphenate hyphenated words: conflicts with underscore
%\savesymbol{BreakableUnderscore}	% suppress BreakableUnderscore defined in hyphenat
									                % (conflicts with underscore)
\usepackage{underscore}	% remove special status of '_' in ordinary text
\usepackage{verbatim}		% improved verbatim environment
\usepackage{parskip}		% handle non-indented paragraphs "properly"
\usepackage{array}			% new column definitions for tables
\usepackage[iso]{isodate} % use iso format for dates
\usepackage{soul}       % strikeouts and underlines for difference markups
\usepackage{color}      % define colors for strikeouts and underlines
\usepackage{amsmath}
\usepackage{mathrsfs}
\usepackage{multicol}
\usepackage{xspace}

\usepackage[T1]{fontenc}
\usepackage{ae}
\usepackage{mathptmx}
\usepackage[scaled=.90]{helvet}

%%--------------------------------------------------
%% Sectioning macros.  
% Each section has a depth, an automatically generated section 
% number, a name, and a short tag.  The depth is an integer in 
% the range [0,5].  (If it proves necessary, it wouldn't take much
% programming to raise the limit from 5 to something larger.)


% The basic sectioning command.  Example:
%    \Sec1[intro.scope]{Scope}
% defines a first-level section whose name is "Scope" and whose short
% tag is intro.scope.  The square brackets are mandatory.
\def\Sec#1[#2]#3{{%
\ifcase#1\let\s=\chapter
      \or\let\s=\section
      \or\let\s=\subsection
      \or\let\s=\subsubsection
      \or\let\s=\paragraph
      \or\let\s=\subparagraph
      \fi%
\s[#3]{#3\hfill[#2]}\relax\label{#2}}}

% A convenience feature (mostly for the convenience of the Project
% Editor, to make it easy to move around large blocks of text):
% the \rSec macro is just like the \Sec macro, except that depths 
% relative to a global variable, SectionDepthBase.  So, for example,
% if SectionDepthBase is 1,
%   \rSec1[temp.arg.type]{Template type arguments}
% is equivalent to
%   \Sec2[temp.arg.type]{Template type arguments}

\newcounter{SectionDepthBase}
\newcounter{scratch}

\def\rSec#1[#2]#3{{%
\setcounter{scratch}{#1}
\addtocounter{scratch}{\value{SectionDepthBase}}
\Sec{\arabic{scratch}}[#2]{#3}}}

% Change the way section headings are formatted.
\renewcommand{\chaptername}{}
\renewcommand{\appendixname}{Annex}

\makeatletter
\def\@makechapterhead#1{%
  \hrule\vspace*{1.5\p@}\hrule
  \vspace*{16\p@}%
  {\parindent \z@ \raggedright \normalfont
    \ifnum \c@secnumdepth >\m@ne
        \huge\bfseries \@chapapp\space \thechapter\space\space\space\space
    \fi
    \interlinepenalty\@M
    \huge \bfseries #1\par\nobreak
  \vspace*{16\p@}%
  \hrule\vspace*{1.5\p@}\hrule
  \vspace*{48\p@}
  }}

\renewcommand\section{\@startsection{section}{1}{0pt}%
                                   {-3.5ex plus -1ex minus -.2ex}%
                                   {.3ex plus .2ex}%
                                   {\normalfont\normalsize\bfseries}}
\renewcommand\section{\@startsection{section}{1}{0pt}%
                                   {2.5ex}% plus 1ex minus .2ex}%
                                   {.3ex}% plus .1ex minus .2 ex}%
                                   {\normalfont\normalsize\bfseries}}

\renewcommand\subsection{\@startsection{subsection}{2}{0pt}%
                                     {-3.25ex plus -1ex minus -.2ex}%
                                     {.3ex plus .2ex}%
                                     {\normalfont\normalsize\bfseries}}

\renewcommand\subsubsection{\@startsection{subsubsection}{3}{0pt}%
                                     {-3.25ex plus -1ex minus -.2ex}%
                                     {.3ex plus .2ex}%
                                     {\normalfont\normalsize\bfseries}}

\renewcommand\paragraph{\@startsection{paragraph}{4}{0pt}%
                                     {-3.25ex plus -1ex minus -.2ex}%
                                     {.3ex \@plus .2ex}%
                                     {\normalfont\normalsize\bfseries}}

\renewcommand\subparagraph{\@startsection{subparagraph}{5}{0pt}%
                                     {-3.25ex plus -1ex minus -.2ex}%
                                     {.3ex plus .2ex}%
                                     {\normalfont\normalsize\bfseries}}
\makeatother

%%--------------------------------------------------
% Heading style for Annexes
\newcommand{\Annex}[3]{\chapter[#2]{\\(#3)\\#2\hfill[#1]}\relax\label{#1}}
\newcommand{\infannex}[2]{\Annex{#1}{#2}{informative}}
\newcommand{\normannex}[2]{\Annex{#1}{#2}{normative}}

\newcommand{\synopsis}[1]{\textbf{#1}}

%%--------------------------------------------------
% General code style
\newcommand{\CodeStyle}{\ttfamily}
\newcommand{\CodeStylex}[1]{\texttt{#1}}

% Code and definitions embedded in text.
\newcommand{\tcode}[1]{\CodeStylex{#1}}
\newcommand{\techterm}[1]{\textit{#1}}

%%--------------------------------------------------
%% allow line break if needed for justification
\newcommand{\brk}{\discretionary{}{}{}}
%  especially for scope qualifier
\newcommand{\colcol}{\brk::\brk}

%%--------------------------------------------------
%% Macros for funky text
\newcommand{\Rplus}{\protect\nolinebreak\hspace{-.07em}\protect\raisebox{.25ex}{\small\textbf{+}}}
\newcommand{\Cpp}{C\Rplus\Rplus}
\newcommand{\opt}{$_\mathit{opt}$}
\newcommand{\shl}{<{<}}
\newcommand{\shr}{>{>}}
\newcommand{\dcr}{-{-}}
\newcommand{\bigohm}[1]{\mathscr{O}(#1)}
\newcommand{\bigoh}[1]{$\bigohm{#1}$\xspace}
\renewcommand{\tilde}{{\smaller$\sim$}\xspace}		% extra level of braces is necessary

%% Notes and examples
\newcommand{\EnterBlock}[1]{[\,\textit{#1:}}
\newcommand{\ExitBlock}[1]{\textit{\ ---\,end #1}\,]}
\newcommand{\enternote}{\EnterBlock{Note}}
\newcommand{\exitnote}{\ExitBlock{note}}
\newcommand{\enterexample}{\EnterBlock{Example}}
\newcommand{\exitexample}{\ExitBlock{example}}

%% Library function descriptions
\newcommand{\Fundescx}[1]{\textit{#1}}
\newcommand{\Fundesc}[1]{\Fundescx{#1:}}
\newcommand{\required}{\Fundesc{Required behavior}}
\newcommand{\requires}{\Fundesc{Requires}}
\newcommand{\effects}{\Fundesc{Effects}}
\newcommand{\postconditions}{\Fundesc{Postconditions}}
\newcommand{\postcondition}{\Fundesc{Postcondition}}
\newcommand{\preconditions}{\Fundesc{Preconditions}}
\newcommand{\precondition}{\Fundesc{Precondition}}
\newcommand{\returns}{\Fundesc{Returns}}
\newcommand{\throws}{\Fundesc{Throws}}
\newcommand{\default}{\Fundesc{Default behavior}}
\newcommand{\complexity}{\Fundesc{Complexity}}
\newcommand{\note}{\Fundesc{Remark}}
\newcommand{\notes}{\Fundesc{Remarks}}
\newcommand{\reallynote}{\Fundesc{Note}}
\newcommand{\reallynotes}{\Fundesc{Notes}}
\newcommand{\implimits}{\Fundesc{Implementation limits}}
\newcommand{\replaceable}{\Fundesc{Replaceable}}
\newcommand{\exceptionsafety}{\Fundesc{Exception safety}}
\newcommand{\returntype}{\Fundesc{Return type}}
\newcommand{\cvalue}{\Fundesc{Value}}
\newcommand{\ctype}{\Fundesc{Type}}

%% Cross reference
\newcommand{\xref}{\textsc{See also:}}

%% NTBS, etc.
\newcommand{\NTS}[1]{\textsc{#1}}
\newcommand{\ntbs}{\NTS{ntbs}}
\newcommand{\ntmbs}{\NTS{ntmbs}}
\newcommand{\ntwcs}{\NTS{ntwcs}}

%% Function argument
\newcommand{\farg}[1]{\texttt{\textit{#1}}}

%% Code annotations
\newcommand{\EXPO}[1]{\textbf{#1}}
\newcommand{\expos}{\EXPO{exposition only}}
\newcommand{\exposr}{\hfill\expos}
\newcommand{\exposrc}{\hfill// \expos}
\newcommand{\impdef}{\EXPO{implementation-defined}}
\newcommand{\notdef}{\EXPO{not defined}}

%% Double underscore
\newcommand{\unun}{\_\,\_}
\newcommand{\xname}[1]{\unun\,#1}
\newcommand{\mname}[1]{\tcode{\unun\,#1\,\unun}}

%% Ranges
\newcommand{\Range}[4]{\tcode{#1\brk{}#3,\brk{}#4\brk{}#2}}
\newcommand{\crange}[2]{\Range{[}{]}{#1}{#2}}
\newcommand{\orange}[2]{\Range{(}{)}{#1}{#2}}
\newcommand{\range}[2]{\Range{[}{)}{#1}{#2}}

%% Unspecified et al
\newcommand{\UNSP}[1]{\textit{#1}}
\newcommand{\unspec}{\UNSP{unspecified}}
\newcommand{\unspecbool}{\UNSP{unspecified-bool-type}}
\newcommand{\unspecnull}{\UNSP{unspecified-null-pointer-type}}
\newcommand{\seebelow}{\UNSP{see below}}
%% Change descriptions
\newcommand{\diffdef}[1]{\hfill\break\textbf{#1:}}
\newcommand{\change}{\diffdef{Change}}
\newcommand{\rationale}{\diffdef{Rationale}}
\newcommand{\effect}{\diffdef{Effect on original feature}}
\newcommand{\difficulty}{\diffdef{Difficulty of converting}}
\newcommand{\howwide}{\diffdef{How widely used}}

%% Difference markups
\definecolor{addclr}{rgb}{0,.4,.4}
\definecolor{remclr}{rgb}{1,0,0}
\newcommand{\added}[1]{\textcolor{addclr}{\ul{#1}}}
\newcommand{\removed}[1]{\textcolor{remclr}{\st{#1}}}
\newcommand{\changed}[2]{\removed{#1}\added{#2}}

%% October, 2005 changes
\newcommand{\addedA}[1]{#1}
\newcommand{\removedA}[1]{}
\newcommand{\changedA}[2]{#2}

%% April, 2006 changes
\newcommand{\addedB}[1]{#1}
\newcommand{\removedB}[1]{}
\newcommand{\changedB}[2]{#2}
\newcommand{\marktr}{}
\newcommand{\ptr}{}
\newcommand{\addedC}[1]{#1}
\newcommand{\removedC}[1]{}
\newcommand{\changedC}[2]{#2}
\newcommand{\additemC}[1]{\item #1}
\newcommand{\addedD}[1]{#1}
\newcommand{\removedD}[1]{}
\newcommand{\changedD}[2]{#2}
\newcommand{\remfootnoteD}[1]{}

%% Miscellaneous
\newcommand{\uniquens}{\textrm{\textit{\textbf{unique}}}}
\newcommand{\stage}[1]{\item{\textbf{Stage #1:}}}

%%--------------------------------------------------
%% Adjust markers
\renewcommand{\thetable}{\arabic{table}}
\renewcommand{\thefigure}{\arabic{figure}}
\renewcommand{\thefootnote}{\arabic{footnote})}

%% Change list item markers from box to dash
\renewcommand{\labelitemi}{---}
\renewcommand{\labelitemii}{---}
\renewcommand{\labelitemiii}{---}
\renewcommand{\labelitemiv}{---}

%%--------------------------------------------------
%% Environments for code listings.

% We use the 'listings' package, with some small customizations.  The
% most interesting customization: all TeX commands are available
% within comments.  Comments are set in italics, keywords and strings
% don't get special treatment.

\lstset{language=C++,
        basicstyle=\CodeStyle\small,
        keywordstyle=,
        stringstyle=,
        xleftmargin=1em,
        showstringspaces=false,
        commentstyle=\itshape\rmfamily,
        columns=flexible,
        keepspaces=true,
        texcl=true}

% Our usual abbreviation for 'listings'.  Comments are in 
% italics.  Arbitrary TeX commands can be used if they're 
% surrounded by @ signs.
\lstnewenvironment{codeblock}
{
 \lstset{escapechar=@}
 \renewcommand{\tcode}[1]{\textup{\CodeStyle##1}}
 \renewcommand{\techterm}[1]{\textit{##1}}
}
{
}

% Permit use of '@' inside codeblock blocks (don't ask)
\makeatletter
\newcommand{\atsign}{@}
\makeatother

%%--------------------------------------------------
%% Paragraph numbering
\newcounter{Paras}
\makeatletter
\@addtoreset{Paras}{chapter}
\@addtoreset{Paras}{section}
\@addtoreset{Paras}{subsection}
\@addtoreset{Paras}{subsubsection}
\@addtoreset{Paras}{paragraph}
\@addtoreset{Paras}{subparagraph}
\def\pnum{\addtocounter{Paras}{1}\noindent\llap{{\footnotesize\arabic{Paras}}\hspace{\@totalleftmargin}\quad}}
\makeatother

% For compatibility only.  We no longer need this environment.
\newenvironment{paras}{}{}

%%--------------------------------------------------
%% Indented text
\newenvironment{indented}
{\list{}{}\item\relax}
{\endlist}

%%--------------------------------------------------
%% Library item descriptions
\lstnewenvironment{itemdecl}
{
 \lstset{escapechar=@,
 xleftmargin=0em,
 aboveskip=2ex,
 belowskip=0ex	% leave this alone: it keeps these things out of the
				% footnote area
 }
}
{
}

\newenvironment{itemdescr}
{
 \begin{indented}}
{
 \end{indented}
}


%%--------------------------------------------------
%% Bnf environments
\newlength{\BnfIndent}
\setlength{\BnfIndent}{\leftmargini}
\newlength{\BnfInc}
\setlength{\BnfInc}{\BnfIndent}
\newlength{\BnfRest}
\setlength{\BnfRest}{2\BnfIndent}
\newcommand{\BnfNontermshape}{\rmfamily\itshape\small}
\newcommand{\BnfTermshape}{\ttfamily\upshape\small}
\newcommand{\nonterminal}[1]{{\BnfNontermshape #1}}

\newenvironment{bnfbase}
 {
 \newcommand{\terminal}[1]{{\BnfTermshape ##1}}
 \newcommand{\descr}[1]{\normalfont{##1}}
 \newcommand{\bnfindentfirst}{\BnfIndent}
 \newcommand{\bnfindentinc}{\BnfInc}
 \newcommand{\bnfindentrest}{\BnfRest}
 \begin{minipage}{.9\hsize}
 \newcommand{\br}{\hfill\\}
 }
 {
 \end{minipage}
 }

\newenvironment{BnfTabBase}[1]
{
 \begin{bnfbase}
 #1
 \begin{indented}
 \begin{tabbing}
 \hspace*{\bnfindentfirst}\=\hspace{\bnfindentinc}\=\hspace{.6in}\=\hspace{.6in}\=\hspace{.6in}\=\hspace{.6in}\=\hspace{.6in}\=\hspace{.6in}\=\hspace{.6in}\=\hspace{.6in}\=\hspace{.6in}\=\hspace{.6in}\=\kill%
}
{
 \end{tabbing}
 \end{indented}
 \end{bnfbase}
}

\newenvironment{bnfkeywordtab}
{
 \begin{BnfTabBase}{\BnfTermshape}
}
{
 \end{BnfTabBase}
}

\newenvironment{bnftab}
{
 \begin{BnfTabBase}{\BnfNontermshape}
}
{
 \end{BnfTabBase}
}

\newenvironment{simplebnf}
{
 \begin{bnfbase}
 \BnfNontermshape
 \begin{indented}
}
{
 \end{indented}
 \end{bnfbase}
}

\newenvironment{bnf}
{
 \begin{bnfbase}
 \list{}
	{
	\setlength{\leftmargin}{\bnfindentrest}
	\setlength{\listparindent}{-\bnfindentinc}
	\setlength{\itemindent}{\listparindent}
	}
 \BnfNontermshape
 \item\relax
}
{
 \endlist
 \end{bnfbase}
}

% non-copied versions of bnf environments
\newenvironment{ncbnftab}
{
 \begin{bnftab}
}
{
 \end{bnftab}
}

\newenvironment{ncsimplebnf}
{
 \begin{simplebnf}
}
{
 \end{simplebnf}
}

\newenvironment{ncbnf}
{
 \begin{bnf}
}
{
 \end{bnf}
}

%%--------------------------------------------------
%% Drawing environment
%
% usage: \begin{drawing}{UNITLENGTH}{WIDTH}{HEIGHT}{CAPTION}
\newenvironment{drawing}[4]
{
\begin{figure}[h]
\setlength{\unitlength}{#1}
\begin{center}
\begin{picture}(#2,#3)\thicklines
}
{
\end{picture}
\end{center}
%\caption{Directed acyclic graph}
\end{figure}
}

%%--------------------------------------------------
%% Table environments

% Base definitions for tables
\newenvironment{TableBase}
{
 \renewcommand{\tcode}[1]{{\CodeStyle##1}}
 \newcommand{\topline}{\hline}
 \newcommand{\capsep}{\hline\hline}
 \newcommand{\rowsep}{\hline}
 \newcommand{\bottomline}{\hline}

%% vertical alignment
 \newcommand{\rb}[1]{\raisebox{1.5ex}[0pt]{##1}}	% move argument up half a row

%% header helpers
 \newcommand{\hdstyle}[1]{\textbf{##1}}				% set header style
 \newcommand{\Head}[3]{\multicolumn{##1}{##2}{\hdstyle{##3}}}	% add title spanning multiple columns
 \newcommand{\lhdrx}[2]{\Head{##1}{|c}{##2}}		% set header for left column spanning #1 columns
 \newcommand{\chdrx}[2]{\Head{##1}{c}{##2}}			% set header for center column spanning #1 columns
 \newcommand{\rhdrx}[2]{\Head{##1}{c|}{##2}}		% set header for right column spanning #1 columns
 \newcommand{\ohdrx}[2]{\Head{##1}{|c|}{##2}}		% set header for only column spanning #1 columns
 \newcommand{\lhdr}[1]{\lhdrx{1}{##1}}				% set header for single left column
 \newcommand{\chdr}[1]{\chdrx{1}{##1}}				% set header for single center column
 \newcommand{\rhdr}[1]{\rhdrx{1}{##1}}				% set header for single right column
 \newcommand{\ohdr}[1]{\ohdrx{1}{##1}}
 \newcommand{\br}{\hfill\break}						% force newline within table entry

%% column styles
 \newcolumntype{x}[1]{>{\raggedright\let\\=\tabularnewline}p{##1}}	% word-wrapped ragged-right
 																	% column, width specified by #1
 \newcolumntype{m}[1]{>{\CodeStyle}l{##1}}							% variable width column, all entries in CodeStyle
}
{
}

% General Usage: TITLE is the title of the table, XREF is the
% cross-reference for the table. LAYOUT is a sequence of column
% type specifiers (e.g. cp{1.0}c), without '|' for the left edge
% or right edge.

% usage: \begin{floattablebase}{TITLE}{XREF}{COLUMNS}{PLACEMENT}
% produces floating table, location determined within limits
% by LaTeX.
\newenvironment{floattablebase}[4]
{
 \begin{TableBase}
 \begin{table}[#4]
 \caption{\label{#2}#1}
 \begin{center}
 \begin{tabular}{|#3|}
}
{
 \bottomline
 \end{tabular}
 \end{center}
 \end{table}
 \end{TableBase}
}

% usage: \begin{floattable}{TITLE}{XREF}{COLUMNS}
% produces floating table, location determined within limits
% by LaTeX.
\newenvironment{floattable}[3]
{
 \begin{floattablebase}{#1}{#2}{#3}{htbp}
}
{
 \end{floattablebase}
}

% usage: \begin{tokentable}{TITLE}{XREF}{HDR1}{HDR2}
% produces six-column table used for lists of replacement tokens;
% the columns are in pairs -- left-hand column has header HDR1,
% right hand column has header HDR2; pairs of columns are separated
% by vertical lines. Used in "trigraph sequences" table in standard.
\newenvironment{tokentable}[4]
{
 \begin{floattablebase}{#1}{#2}{cc|cc|cc}{htbp}
 \topline
 \textit{#3}   &   \textit{#4}    &
 \textit{#3}   &   \textit{#4}    &
 \textit{#3}   &   \textit{#4}    \\ \capsep
}
{
 \end{floattablebase}
}

% usage: \begin{libsumtabase}{TITLE}{XREF}{HDR1}{HDR2}
% produces two-column table with column headers HDR1 and HDR2.
% Used in "Library Categories" table in standard, and used as
% base for other library summary tables.
\newenvironment{libsumtabbase}[4]
{
 \begin{floattable}{#1}{#2}{ll}
 \topline
 \lhdr{#3}	&	\hdstyle{#4}	\\ \capsep
}
{
 \end{floattable}
}

% usage: \begin{libsumtab}{TITLE}{XREF}
% produces two-column table with column headers "Subclause" and "Header(s)".
% Used in "C++ Headers for Freestanding Implementations" table in standard.
\newenvironment{libsumtab}[2]
{
 \begin{libsumtabbase}{#1}{#2}{Subclause}{Header(s)}
}
{
 \end{libsumtabbase}
}

% usage: \begin{LibSynTab}{CAPTION}{TITLE}{XREF}{COUNT}{LAYOUT}
% produces table with COUNT columns. Used as base for
% C library description tables
\newcounter{LibSynTabCols}
\newcounter{LibSynTabWd}
\newenvironment{LibSynTabBase}[5]
{
 \setcounter{LibSynTabCols}{#4}
 \setcounter{LibSynTabWd}{#4}
 \addtocounter{LibSynTabWd}{-1}
 \newcommand{\centry}[1]{\textbf{##1}:}
 \newcommand{\macro}{\centry{Macro}}
 \newcommand{\macros}{\centry{Macros}}
 \newcommand{\function}{\centry{Function}}
 \newcommand{\functions}{\centry{Functions}}
 \newcommand{\templates}{\centry{Templates}}
 \newcommand{\type}{\centry{Type}}
 \newcommand{\types}{\centry{Types}}
 \newcommand{\values}{\centry{Values}}
 \newcommand{\struct}{\centry{Struct}}
 \newcommand{\cspan}[1]{\multicolumn{\value{LibSynTabCols}}{|l|}{##1}}
 \begin{floattable}{#1 \tcode{<#2>}\ synopsis}{#3}
 {#5}
 \topline
 \lhdr{Type}	&	\rhdrx{\value{LibSynTabWd}}{Name(s)}	\\ \capsep
}
{
 \end{floattable}
}

% usage: \begin{LibSynTab}{TITLE}{XREF}{COUNT}{LAYOUT}
% produces table with COUNT columns. Used as base for description tables
% for C library
\newenvironment{LibSynTab}[4]
{
 \begin{LibSynTabBase}{Header}{#1}{#2}{#3}{#4}
}
{
 \end{LibSynTabBase}
}

% usage: \begin{LibSynTabAdd}{TITLE}{XREF}{COUNT}{LAYOUT}
% produces table with COUNT columns. Used as base for description tables
% for additions to C library
\newenvironment{LibSynTabAdd}[4]
{
 \begin{LibSynTabBase}{Additions to header}{#1}{#2}{#3}{#4}
}
{
 \end{LibSynTabBase}
}

% usage: \begin{libsyntabN}{TITLE}{XREF}
%        \begin{libsyntabaddN}{TITLE}{XREF}
% produces a table with N columns for C library description tables
\newenvironment{libsyntab2}[2]
{
 \begin{LibSynTab}{#1}{#2}{2}{ll}
}
{
 \end{LibSynTab}
}

\newenvironment{libsyntab3}[2]
{
 \begin{LibSynTab}{#1}{#2}{3}{lll}
}
{
 \end{LibSynTab}
}

\newenvironment{libsyntab4}[2]
{
 \begin{LibSynTab}{#1}{#2}{4}{llll}
}
{
 \end{LibSynTab}
}

\newenvironment{libsyntab5}[2]
{
 \begin{LibSynTab}{#1}{#2}{5}{lllll}
}
{
 \end{LibSynTab}
}

\newenvironment{libsyntab6}[2]
{
 \begin{LibSynTab}{#1}{#2}{6}{llllll}
}
{
 \end{LibSynTab}
}

\newenvironment{libsyntabadd2}[2]
{
 \begin{LibSynTabAdd}{#1}{#2}{2}{ll}
}
{
 \end{LibSynTabAdd}
}

\newenvironment{libsyntabadd3}[2]
{
 \begin{LibSynTabAdd}{#1}{#2}{3}{lll}
}
{
 \end{LibSynTabAdd}
}

\newenvironment{libsyntabadd4}[2]
{
 \begin{LibSynTabAdd}{#1}{#2}{4}{llll}
}
{
 \end{LibSynTabAdd}
}

\newenvironment{libsyntabadd5}[2]
{
 \begin{LibSynTabAdd}{#1}{#2}{5}{lllll}
}
{
 \end{LibSynTabAdd}
}

\newenvironment{libsyntabadd6}[2]
{
 \begin{LibSynTabAdd}{#1}{#2}{6}{llllll}
}
{
 \end{LibSynTabAdd}
}

% usage: \begin{LongTable}{TITLE}{XREF}{LAYOUT}
% produces table that handles page breaks sensibly.
\newenvironment{LongTable}[3]
{
 \begin{TableBase}
 \begin{longtable}
 {|#3|}\caption{#1}\label{#2}
}
{
 \bottomline
 \end{longtable}
 \end{TableBase}
}

% usage: \begin{twocol}{TITLE}{XREF}
% produces a two-column breakable table. Used in
% "simple-type-specifiers and the types they specify" in the standard.
\newenvironment{twocol}[2]
{
 \begin{LongTable}
 {#1}{#2}
 {ll}
}
{
 \end{LongTable}
}

% usage: \begin{libreqtabN}{TITLE}{XREF}
% produces an N-column brekable table. Used in
% most of the library clauses for requirements tables.
% Example at "Position type requirements" in the standard.

\newenvironment{libreqtab1}[2]
{
 \begin{LongTable}
 {#1}{#2}
 {x{.55\hsize}}
}
{
 \end{LongTable}
}

\newenvironment{libreqtab2}[2]
{
 \begin{LongTable}
 {#1}{#2}
 {lx{.55\hsize}}
}
{
 \end{LongTable}
}

\newenvironment{libreqtab2a}[2]
{
 \begin{LongTable}
 {#1}{#2}
 {x{.30\hsize}x{.68\hsize}}
}
{
 \end{LongTable}
}

\newenvironment{libreqtab3}[2]
{
 \begin{LongTable}
 {#1}{#2}
 {x{.28\hsize}x{.18\hsize}x{.43\hsize}}
}
{
 \end{LongTable}
}

\newenvironment{libreqtab3a}[2]
{
 \begin{LongTable}
 {#1}{#2}
 {x{.28\hsize}x{.33\hsize}x{.29\hsize}}
}
{
 \end{LongTable}
}

\newenvironment{libreqtab3b}[2]
{
 \begin{LongTable}
 {#1}{#2}
 {x{.40\hsize}x{.25\hsize}x{.25\hsize}}
}
{
 \end{LongTable}
}

\newenvironment{libreqtab3c}[2]
{
 \begin{LongTable}
 {#1}{#2}
 {x{.30\hsize}x{.25\hsize}x{.35\hsize}}
}
{
 \end{LongTable}
}

\newenvironment{libreqtab3d}[2]
{
 \begin{LongTable}
 {#1}{#2}
 {x{.32\hsize}x{.27\hsize}x{.36\hsize}}
}
{
 \end{LongTable}
}

\newenvironment{libreqtab3e}[2]
{
 \begin{LongTable}
 {#1}{#2}
 {x{.38\hsize}x{.27\hsize}x{.25\hsize}}
}
{
 \end{LongTable}
}

\newenvironment{libreqtab3f}[2]
{
 \begin{LongTable}
 {#1}{#2}
 {x{.40\hsize}x{.22\hsize}x{.31\hsize}}
}
{
 \end{LongTable}
}

\newenvironment{libreqtab4}[2]
{
 \begin{LongTable}
 {#1}{#2}
}
{
 \end{LongTable}
}

\newenvironment{libreqtab4a}[2]
{
 \begin{LongTable}
 {#1}{#2}
 {x{.14\hsize}x{.30\hsize}x{.30\hsize}x{.14\hsize}}
}
{
 \end{LongTable}
}

\newenvironment{libreqtab4b}[2]
{
 \begin{LongTable}
 {#1}{#2}
 {x{.13\hsize}x{.15\hsize}x{.29\hsize}x{.27\hsize}}
}
{
 \end{LongTable}
}

\newenvironment{libreqtab4c}[2]
{
 \begin{LongTable}
 {#1}{#2}
 {x{.16\hsize}x{.21\hsize}x{.21\hsize}x{.30\hsize}}
}
{
 \end{LongTable}
}

\newenvironment{libreqtab4d}[2]
{
 \begin{LongTable}
 {#1}{#2}
 {x{.22\hsize}x{.22\hsize}x{.30\hsize}x{.15\hsize}}
}
{
 \end{LongTable}
}

\newenvironment{libreqtab5}[2]
{
 \begin{LongTable}
 {#1}{#2}
 {x{.14\hsize}x{.14\hsize}x{.20\hsize}x{.20\hsize}x{.14\hsize}}
}
{
 \end{LongTable}
}

% usage: \begin{libtab2}{TITLE}{XREF}{LAYOUT}{HDR1}{HDR2}
% produces two-column table with column headers HDR1 and HDR2.
% Used in "seekoff positioning" in the standard.
\newenvironment{libtab2}[5]
{
 \begin{floattable}
 {#1}{#2}{#3}
 \topline
 \lhdr{#4}	&	\rhdr{#5}	\\ \capsep
}
{
 \end{floattable}
}

% usage: \begin{longlibtab2}{TITLE}{XREF}{LAYOUT}{HDR1}{HDR2}
% produces two-column table with column headers HDR1 and HDR2.
\newenvironment{longlibtab2}[5]
{
 \begin{LongTable}{#1}{#2}{#3}
 \\ \topline
 \lhdr{#4}	&	\rhdr{#5}	\\ \capsep
}
{
  \end{LongTable}
}

% usage: \begin{LibEffTab}{TITLE}{XREF}{HDR2}{WD2}
% produces a two-column table with left column header "Element"
% and right column header HDR2, right column word-wrapped with
% width specified by WD2.
\newenvironment{LibEffTab}[4]
{
 \begin{libtab2}{#1}{#2}{lp{#4}}{Element}{#3}
}
{
 \end{libtab2}
}

% Same as LibEffTab except that it uses a long table.
\newenvironment{longLibEffTab}[4]
{
 \begin{longlibtab2}{#1}{#2}{lp{#4}}{Element}{#3}
}
{
 \end{longlibtab2}
}

% usage: \begin{libefftab}{TITLE}{XREF}
% produces a two-column effects table with right column
% header "Effect(s) if set", width 4.5 in. Used in "fmtflags effects"
% table in standard.
\newenvironment{libefftab}[2]
{
 \begin{LibEffTab}{#1}{#2}{Effect(s) if set}{4.5in}
}
{
 \end{LibEffTab}
}

% Same as libefftab except that it uses a long table.
\newenvironment{longlibefftab}[2]
{
 \begin{longLibEffTab}{#1}{#2}{Effect(s) if set}{4.5in}
}
{
 \end{longLibEffTab}
}

% usage: \begin{libefftabmean}{TITLE}{XREF}
% produces a two-column effects table with right column
% header "Meaning", width 4.5 in. Used in "seekdir effects"
% table in standard.
\newenvironment{libefftabmean}[2]
{
 \begin{LibEffTab}{#1}{#2}{Meaning}{4.5in}
}
{
 \end{LibEffTab}
}

% Same as libefftabmean except that it uses a long table.
\newenvironment{longlibefftabmean}[2]
{
 \begin{longLibEffTab}{#1}{#2}{Meaning}{4.5in}
}
{
 \end{longLibEffTab}
}

% usage: \begin{libefftabvalue}{TITLE}{XREF}
% produces a two-column effects table with right column
% header "Value", width 3 in. Used in "basic_ios::init() effects"
% table in standard.
\newenvironment{libefftabvalue}[2]
{
 \begin{LibEffTab}{#1}{#2}{Value}{3in}
}
{
 \end{LibEffTab}
}

% Same as libefftabvalue except that it uses a long table and a
% slightly wider column.
\newenvironment{longlibefftabvalue}[2]
{
 \begin{longLibEffTab}{#1}{#2}{Value}{3.5in}
}
{
 \end{longLibEffTab}
}

% usage: \begin{liberrtab}{TITLE}{XREF} produces a two-column table
% with left column header ``Value'' and right header "Error
% condition", width 4.5 in. Used in regex clause in the TR.

\newenvironment{liberrtab}[2]
{
 \begin{libtab2}{#1}{#2}{lp{4.5in}}{Value}{Error condition}
}
{
 \end{libtab2}
}

% Like liberrtab except that it uses a long table.
\newenvironment{longliberrtab}[2]
{
 \begin{longlibtab2}{#1}{#2}{lp{4.5in}}{Value}{Error condition}
}
{
 \end{longlibtab2}
}

% enumerate with lowercase letters
\newenvironment{enumeratea}
{
 \renewcommand{\labelenumi}{\alph{enumi})}
 \begin{enumerate}
}
{
 \end{enumerate}
}

% enumerate with arabic numbers
\newenvironment{enumeraten}
{
 \renewcommand{\labelenumi}{\arabic{enumi})}
 \begin{enumerate}
}
{
 \end{enumerate}
}

%%--------------------------------------------------
%% Definitions section
% usage: \definition{name}{xref}
\newcommand{\definition}[2]{\rSec2[#2]{#1}}
% for ISO format, use:
%\newcommand{\definition}[2]
% {\hfill\\\vspace{.25ex plus .5ex minus .2ex}\\
% \addtocounter{subsection}{1}%
% \textbf{\thesubsection\hfill\relax[#2]}\\
% \textbf{#1}\label{#2}\\
% }
%----------------------------------------------------
% Doug's macros

% For editorial comments.
\definecolor{editbackground}{rgb}{.8,.8,.8}
\newcommand{\einline}[1]{\colorbox{editbackground}{#1}}
\newcommand{\editorial}[1]{\colorbox{editbackground}{\begin{minipage}{\linewidth}#1\end{minipage}}}
% For editorial comments as footnotes.
\newcommand{\efootnote}[1]{\footnote{\editorial{#1}}}
% For editorial "remove this!" comments
\newcommand{\eremove}[1]{\textcolor{remclr}{[[\textbf{#1}]]}}
% Macros that note precisely what we're adding, removing, or changing
% within a paragraph.
\newcommand{\addedConcepts}[1]{\added{#1}}
\newcommand{\removedConcepts}[1]{\removed{#1}}
\newcommand{\changedConcepts}[2]{\changed{#1}{#2}}

%% Concepts revision 4 changes
\definecolor{ccadd}{rgb}{0,.6,0}
\newcommand{\addedCC}[1]{\textcolor{ccadd}{\ul{#1}}}
\newcommand{\removedCC}[1]{\textcolor{remclr}{\st{#1}}}
\newcommand{\changedCC}[2]{\removedCC{#1}\addedCC{#2}}
\newcommand{\changedCCC}[2]{\textcolor{ccadd}{\st{#1}}\addedCC{#2}}
\newcommand{\removedCCC}[1]{\textcolor{ccadd}{\st{#1}}}
\newcommand{\remitemCC}[1]{\remitem{#1}}
\newcommand{\additemCC}[1]{\item\addedCC{#1}}

%% Concepts changes for the next revision
\definecolor{zadd}{rgb}{0.8,0,0.8}
\newcommand{\addedZ}[1]{\textcolor{zadd}{\ul{#1}}}
\newcommand{\removedZ}[1]{\textcolor{remclr}{\st{#1}}}
\newcommand{\changedCZ}[2]{\textcolor{addclr}{\st{#1}}\addedZ{#2}}

% Text common to all of the library proposals
\newcommand{\libintrotext}[1]{This document proposes changes to #1 of
  the \Cpp{} Standard Library in order to make full use of
  concepts~\cite{GregorStroustrup07:concepts_wording_rev_5}. We make
  every attempt to provide complete backward compatibility with the
  pre-concept Standard Library, and note each place where we have
  knowingly changed semantics.

This document is formatted in the same manner as the latest working draft of
the \Cpp{} standard (N2588). Future versions of this document will
track the working draft and the concepts proposal as they
evolve. Wherever the numbering of a (sub)section matches a section of
the working paper, the text in this document should be considered
replacement text, unless editorial comments state otherwise. All
editorial comments will \einline{have a gray background}.  Changes to
the replacement text are categorized and typeset as
\addedConcepts{additions}, \removedConcepts{removals}, or
\changedConcepts{changes}{modifications}.}

%%--------------------------------------------------
%% PDF

\usepackage[pdftex,
            pdftitle={Concepts for the C++0x Standard Library: Iterators},
            pdfsubject={C++ International Standard Proposal},
            pdfcreator={Douglas Gregor},
            bookmarks=true,
            bookmarksnumbered=true,
            pdfpagelabels=true,
            pdfpagemode=UseOutlines,
            pdfstartview=FitH,
            linktocpage=true,
            colorlinks=true,
            linkcolor=blue,
            plainpages=false
           ]{hyperref}

%%--------------------------------------------------
%% Set section numbering limit, toc limit
\setcounter{secnumdepth}{4}
\setcounter{tocdepth}{1}

%%--------------------------------------------------
%% Parameters that govern document appearance
\setlength{\oddsidemargin}{0pt}
\setlength{\evensidemargin}{0pt}
\setlength{\textwidth}{6.6in}

\newcommand{\resetcolor}{\textcolor{addclr}{}}

%%--------------------------------------------------
%% Handle special hyphenation rules
\hyphenation{tem-plate ex-am-ple in-put-it-er-a-tor}

% Do not put blank pages after chapters that end on odd-numbered pages.
\def\cleardoublepage{\clearpage\if@twoside%
  \ifodd\c@page\else\hbox{}\thispagestyle{empty}\newpage%
  \if@twocolumn\hbox{}\newpage\fi\fi\fi}

\begin{document}
\raggedbottom

\begin{titlepage}
\begin{center}
\huge
Concepts for the C++0x Standard Library: Iterators \\
(Revision 4)
\vspace{0.5in}

\normalsize
Douglas Gregor and Andrew Lumsdaine \\
\href{mailto:dgregor@osl.iu.edu}{dgregor@osl.iu.edu}, \href{mailto:lums@osl.iu.edu}{lums@osl.iu.edu}
\end{center}

\vspace{1in}
\par\noindent Document number: DRAFT \vspace{-6pt}
\par\noindent Revises document number: N2734=08-0244 \vspace{-6pt}
\par\noindent Date: \today\vspace{-6pt}
\par\noindent Project: Programming Language \Cpp{}, Library Working Group\vspace{-6pt}
\par\noindent Reply-to: Douglas Gregor $<$\href{mailto:dgregor@osl.iu.edu}{dgregor@osl.iu.edu}$>$\vspace{-6pt}

\section*{Introduction}
\libintrotext{24}

\paragraph*{Changes from N2734}
\begin{itemize}
\item Fixed the \tcode{HasMinus} requirement on
  \tcode{reverse_iterator}'s \tcode{operator-}.
\item When using \tcode{decltype} in the return type of one of the
  iterator adaptors' \tcode{operator-} operations, use the
  \tcode{base()} function rather than \tcode{current} to retrieve the
  underlying iterator.
\end{itemize}
\end{titlepage}

%% --------------------------------------------------
%% Headers and footers
\pagestyle{fancy}
\fancyhead[LE,RO]{\textbf{\rightmark}}
\fancyhead[RE]{\textbf{\leftmark\hspace{1em}\thepage}}
\fancyhead[LO]{\textbf{\thepage\hspace{1em}\leftmark}}
\fancyfoot[C]{Draft}

\fancypagestyle{plain}{
\renewcommand{\headrulewidth}{0in}
\fancyhead[LE,RO]{}
\fancyhead[RE,LO]{}
\fancyfoot{}
}

\renewcommand{\sectionmark}[1]{\markright{\thesection\hspace{1em}#1}}
\renewcommand{\chaptermark}[1]{\markboth{#1}{}}

\setcounter{chapter}{23}
\rSec0[iterators]{Iterators library}

\begin{paras}

\setcounter{Paras}{1}

\textcolor{black}{\pnum}
The following subclauses describe
iterator \changedConcepts{requirements}{concepts}, and
components for
iterator primitives,
predefined iterators,
and stream iterators,
as summarized in Table~\ref{tab:iterators.lib.summary}.

\begin{libsumtab}{Iterators library summary}{tab:iterators.lib.summary}
\ref{iterator.concepts} \changedConcepts{Requirements}{Concepts}            &      \addedConcepts{\tt <iterator_concepts>}                                   \\ \rowsep
\ref{depr.lib.iterator.primitives} Iterator primitives       &       \tcode{<iterator>}              \\
\ref{predef.iterators} Predefined iterators         &                                                       \\
\ref{stream.iterators} Stream iterators                     &                                                       \\
\end{libsumtab}

\rSec1[iterator.concepts]{Iterator concepts}
\editorial{The proposed wording for this section is in the separate proposal, ``Iterator Concepts for the C++0x Standard Library''.}

\rSec1[iterator.synopsis]{Header \tcode{<iterator>}\ synopsis}

\index{iterator@\tcode{<iterator>}}%
\begin{codeblock}
namespace std {
  // \ref{depr.lib.iterator.primitives}, primitives:
  template<class Iterator> struct iterator_traits;
  template<class T> struct iterator_traits<T*>;

  template<class Category, class T, class Distance = ptrdiff_t,
       class Pointer = T*, class Reference = T&> struct iterator;

  struct input_iterator_tag { };
  struct output_iterator_tag { };
  struct forward_iterator_tag: public input_iterator_tag { };
  struct bidirectional_iterator_tag: public forward_iterator_tag { };
  struct random_access_iterator_tag: public bidirectional_iterator_tag { };

  // \ref{iterator.operations}, iterator operations:
  template <@\changedConcepts{class InputIterator}{InputIterator Iter}@@\removedConcepts{, class Distance}@>
    void advance(@\changedConcepts{InputIterator}{Iter}@& i, @\changedConcepts{Distance}{Iter::difference_type}@ n);
  @\addedConcepts{template <BidirectionalIterator Iter>}@
    @\addedConcepts{void advance(Iter\& i, Iter::difference_type n);}@
  @\addedConcepts{template <RandomAccessIterator Iter>}@
    @\addedConcepts{void advance(Iter\& i, Iter::difference_type n);}@
  template <@\changedConcepts{class InputIterator}{InputIterator Iter}@>
    @\changedConcepts{typename iterator_traits<InputIterator>::difference_type}{Iter::difference_type}@
    distance(@\changedConcepts{InputIterator}{Iter}@ first, @\changedConcepts{InputIterator}{Iter}@ last);
  @\addedConcepts{template <RandomAccessIterator Iter>}@
    @\addedConcepts{Iter::difference_type}@
    @\addedConcepts{distance(Iter first, Iter last);}@
  template <@\changedConcepts{class InputIterator}{InputIterator Iter}@>
    @\changedConcepts{InputIterator}{Iter}@ next(@\changedConcepts{InputIterator}{Iter}@ x,
      @\changedConcepts{typename std::iterator_traits<InputIterator>::difference_type}{Iter::difference_type}@ n = 1);
  template <@\changedConcepts{class BidirectionalIterator}{BidirectionalIterator Iter}@>
    @\changedConcepts{BidirectionalIterator}{Iter}@ prev(@\changedConcepts{BidirectionalIterator}{Iter}@ x,
      @\changedConcepts{typename std::iterator_traits<BidirectionalIterator>::difference_type}{Iter::difference_type}@ n = 1);

  // \ref{predef.iterators}, predefined iterators:
  template <@\changedConcepts{class}{BidirectionalIterator}@ Iter@\removedConcepts{ator}@> class reverse_iterator;

  template <@\changedConcepts{class}{BidirectionalIterator}@ Iter@\removedConcepts{ator}@1, @\changedConcepts{class}{BidirectionalIterator}@ Iter@\removedConcepts{ator}@2>
    @\addedConcepts{requires HasEqualTo<Iter1, Iter2>}@
    bool operator==(
      const reverse_iterator<Iter@\removedConcepts{ator}@1>& x,
      const reverse_iterator<Iter@\removedConcepts{ator}@2>& y);
  template <@\changedConcepts{class}{RandomAccessIterator}@ Iter@\removedConcepts{ator}@1, @\changedConcepts{class}{RandomAccessIterator}@ Iter@\removedConcepts{ator}@2>
    @\addedConcepts{requires HasGreater<Iter1, Iter2>}@
    bool operator<(
      const reverse_iterator<Iter@\removedConcepts{ator}@1>& x,
      const reverse_iterator<Iter@\removedConcepts{ator}@2>& y);
  template <@\changedConcepts{class}{BidirectionalIterator}@ Iter@\removedConcepts{ator}@1, @\changedConcepts{class}{BidirectionalIterator}@ Iter@\removedConcepts{ator}@2>
    @\addedConcepts{requires HasNotEqualTo<Iter1, Iter2>}@
    bool operator!=(
      const reverse_iterator<Iter@\removedConcepts{ator}@1>& x,
      const reverse_iterator<Iter@\removedConcepts{ator}@2>& y);
  template <@\changedConcepts{class}{RandomAccessIterator}@ Iter@\removedConcepts{ator}@1, @\changedConcepts{class}{RandomAccessIterator}@ Iter@\removedConcepts{ator}@2>
    @\addedConcepts{requires HasLess<Iter1, Iter2>}@
    bool operator>(
      const reverse_iterator<Iter@\removedConcepts{ator}@1>& x,
      const reverse_iterator<Iter@\removedConcepts{ator}@2>& y);
  template <@\changedConcepts{class}{RandomAccessIterator}@ Iter@\removedConcepts{ator}@1, @\changedConcepts{class}{RandomAccessIterator}@ Iter@\removedConcepts{ator}@2>
    @\addedConcepts{requires HasLessEqual<Iter1, Iter2>}@
    bool operator>=(
      const reverse_iterator<Iter@\removedConcepts{ator}@1>& x,
      const reverse_iterator<Iter@\removedConcepts{ator}@2>& y);
  template <@\changedConcepts{class}{RandomAccessIterator}@ Iter@\removedConcepts{ator}@1, @\changedConcepts{class}{RandomAccessIterator}@ Iter@\removedConcepts{ator}@2>
    @\addedConcepts{requires HasGreaterEqual<Iter1, Iter2>}@
    bool operator<=(
      const reverse_iterator<Iter@\removedConcepts{ator}@1>& x,
      const reverse_iterator<Iter@\removedConcepts{ator}@2>& y);
  template <@\changedConcepts{class}{RandomAccessIterator}@ Iter@\removedConcepts{ator}@1, @\changedConcepts{class}{RandomAccessIterator}@ Iter@\removedConcepts{ator}@2>
    @\addedConcepts{requires HasMinus<Iter2, Iter1>}@
    auto operator-(
      const reverse_iterator<Iter@\removedConcepts{ator}@1>& x,
      const reverse_iterator<Iter@\removedConcepts{ator}@2>& y) -> decltype(y.base() - x.base());
  template <@\changedConcepts{class}{RandomAccessIterator}@ Iterator>
    reverse_iterator<Iter@\removedConcepts{ator}@> operator+(
      @\changedConcepts{typename reverse_iterator<Iterator>::difference_type}{Iter::difference_type}@ n,
      const reverse_iterator<Iter@\removedConcepts{ator}@>& x);

  @\addedConcepts{template<BidirectionalIterator Iter>}@
  @\addedConcepts{concept_map BidirectionalIterator<reverse_iterator<Iter> > \{ \}}@

  @\addedConcepts{template<RandomAccessIterator Iter>}@
  @\addedConcepts{concept_map RandomAccessIterator<reverse_iterator<Iter> > \{ \}}@

  template <@\changedConcepts{class}{BackInsertionContainer}@ Cont@\removedConcepts{ainer}@> class back_insert_iterator;
  template <@\changedConcepts{class}{BackInsertionContainer}@ Cont@\removedConcepts{ainer}@>
    back_insert_iterator<Cont@\removedConcepts{ainer}@> back_inserter(Cont@\removedConcepts{ainer}@& x);
  @\addedConcepts{template<BackInsertionContainer Cont>}@
    @\addedConcepts{concept_map Iterator<back_insert_iterator<Cont> > \{ \}}@

  template <@\changedConcepts{class}{FrontInsertionContainer}@ Cont@\removedConcepts{ainer}@> class front_insert_iterator;
  template <@\changedConcepts{class}{FrontInsertionContainer}@ Cont@\removedConcepts{ainer}@>
    front_insert_iterator<Cont@\removedConcepts{ainer}@> front_inserter(Cont@\removedConcepts{ainer}@& x);
  @\addedConcepts{template<FrontInsertionContainer Cont>}@
    @\addedConcepts{concept_map Iterator<front_insert_iterator<Cont> > \{ \}}@

  template <@\changedConcepts{class}{InsertionContainer}@ Cont@\removedConcepts{ainer}@> class insert_iterator;
  template <@\changedConcepts{class}{InsertionContainer}@ Cont@\removedConcepts{ainer}@>
    insert_iterator<Cont@\removedConcepts{ainer}@> inserter(Cont@\removedConcepts{ainer}@& x, Cont@\removedConcepts{ainer}@::iterator i);
  @\addedConcepts{template<InsertionContainer Cont>}@
    @\addedConcepts{concept_map Iterator<insert_iterator<Cont> > \{ \}}@

  template <@\changedConcepts{class}{InputIterator}@ Iter@\removedConcepts{ator}@> class move_iterator;
  template <@\changedConcepts{class}{InputIterator}@ Iter@\removedConcepts{ator}@1, @\changedConcepts{class}{InputIterator}@ Iter@\removedConcepts{ator}@2>
    @\addedConcepts{requires HasEqualTo<Iter1, Iter2>}@
    bool operator==(
      const move_iterator<Iter@\removedConcepts{ator}@1>& x, const move_iterator<Iter@\removedConcepts{ator}@2>& y);
  template <@\changedConcepts{class}{InputIterator}@ Iter@\removedConcepts{ator}@1, @\changedConcepts{class}{InputIterator}@ Iter@\removedConcepts{ator}@2>
    @\addedConcepts{requires HasEqualTo<Iter1, Iter2>}@
    bool operator!=(
      const move_iterator<Iter@\removedConcepts{ator}@1>& x, const move_iterator<Iter@\removedConcepts{ator}@2>& y);
  template <@\changedConcepts{class}{RandomAccessIterator}@ Iter@\removedConcepts{ator}@1, @\changedConcepts{class}{RandomAccessIterator}@ Iter@\removedConcepts{ator}@2>
    @\addedConcepts{requires HasLess<Iter1, Iter2>}@
    bool operator<(
      const move_iterator<Iter@\removedConcepts{ator}@1>& x, const move_iterator<Iter@\removedConcepts{ator}@2>& y);
  template <@\changedConcepts{class}{RandomAccessIterator}@ Iter@\removedConcepts{ator}@1, @\changedConcepts{class}{RandomAccessIterator}@ Iter@\removedConcepts{ator}@2>
    @\addedConcepts{requires HasLess<Iter2, Iter1>}@
    bool operator<=(
      const move_iterator<Iter@\removedConcepts{ator}@1>& x, const move_iterator<Iter@\removedConcepts{ator}@2>& y);
  template <@\changedConcepts{class}{RandomAccessIterator}@ Iter@\removedConcepts{ator}@1, @\changedConcepts{class}{RandomAccessIterator}@ Iter@\removedConcepts{ator}@2>
    @\addedConcepts{requires HasLess<Iter2, Iter1>}@
    bool operator>(
      const move_iterator<Iter@\removedConcepts{ator}@1>& x, const move_iterator<Iter@\removedConcepts{ator}@2>& y);
  template <@\changedConcepts{class}{RandomAccessIterator}@ Iter@\removedConcepts{ator}@1, @\changedConcepts{class}{RandomAccessIterator}@ Iter@\removedConcepts{ator}@2>
    @\addedConcepts{requires HasLess<Iter1, Iter2>}@
    bool operator>=(
      const move_iterator<Iter@\removedConcepts{ator}@1>& x, const move_iterator<Iter@\removedConcepts{ator}@2>& y);

  template <@\changedConcepts{class}{RandomAccessIterator}@ Iter@\removedConcepts{ator}@1, @\changedConcepts{class}{RandomAccessIterator}@ Iter@\removedConcepts{ator}@2>
    @\addedConcepts{requires HasMinus<Iter1, Iter2>}@
    auto operator-(
      const move_iterator<Iter@\removedConcepts{ator}@1>& x, 
      const move_iterator<Iter@\removedConcepts{ator}@2>& y) -> decltype(x.base() - y.base());
  template <@\changedConcepts{class}{RandomAccessIterator}@ Iterator>
    move_iterator<Iterator> operator+(
      @\changedConcepts{typename move_iterator<Iterator>}{Iter}@::difference_type n, const move_iterator<Iterator>& x);
  template <@\changedConcepts{class}{InputIterator}@ Iter@\removedConcepts{ator}@>
    move_iterator<Iter@\removedConcepts{ator}@> make_move_iterator(const Iterator& i);
  @\addedConcepts{template<InputIterator Iter>}@
    @\addedConcepts{concept_map InputIterator<move_iterator<Iter> > \{ \}}@
  @\addedConcepts{template<ForwardIterator Iter>}@
    @\addedConcepts{concept_map ForwardIterator<move_iterator<Iter> > \{ \}}@
  @\addedConcepts{template<BidirectionalIterator Iter>}@
    @\addedConcepts{concept_map BidirectionalIterator<move_iterator<Iter> > \{ \}}@
  @\addedConcepts{template<RandomAccessIterator Iter>}@
    @\addedConcepts{concept_map RandomAccessIterator<move_iterator<Iter> > \{ \}}@

  // \ref{stream.iterators}, stream iterators:
  template <class T, class charT = char, class traits = char_traits<charT>,
      class Distance = ptrdiff_t>
  class istream_iterator;
  template <class T, class charT, class traits, class Distance>
    bool operator==(const istream_iterator<T,charT,traits,Distance>& x,
            const istream_iterator<T,charT,traits,Distance>& y);
  template <class T, class charT, class traits, class Distance>
    bool operator!=(const istream_iterator<T,charT,traits,Distance>& x,
            const istream_iterator<T,charT,traits,Distance>& y);

  template <class T, class charT = char, class traits = char_traits<charT> >
      class ostream_iterator;

  template<class charT, class traits = char_traits<charT> >
    class istreambuf_iterator;
  template <class charT, class traits>
    bool operator==(const istreambuf_iterator<charT,traits>& @\farg{a}@,
            const istreambuf_iterator<charT,traits>& @\farg{b}@);
  template <class charT, class traits>
    bool operator!=(const istreambuf_iterator<charT,traits>& @\farg{a}@,
            const istreambuf_iterator<charT,traits>& @\farg{b}@);

  template <class charT, class traits = char_traits<charT> >
    class ostreambuf_iterator;
}
\end{codeblock}

\rSec1[iterator.primitives]{Iterator primitives}
\setcounter{subsection}{3}
\rSec2[iterator.operations]{Iterator operations}

\pnum
Since only random access iterators provide
\tcode{+}\
and
\tcode{-}\
operators, the library provides two
function templates
\tcode{advance}\
and
\tcode{distance}.
These
function templates
use
\tcode{+}\
and
\tcode{-}\
for random access iterators (and are, therefore, constant
time for them); for input, forward and bidirectional iterators they use
\tcode{++}\
to provide linear time
implementations.

\index{advance@\tcode{advance}}%
\begin{itemdecl}
template <@\changedConcepts{class InputIterator}{InputIterator Iter}@@\removedConcepts{, class Distance}@>
  void advance(@\changedConcepts{InputIterator}{Iter}@& i, @\changedConcepts{Distance}{Iter::difference_type}@ n);
@\addedConcepts{template <BidirectionalIterator Iter>}@
  @\addedConcepts{void advance(Iter\& i, Iter::difference_type n);}@
@\addedConcepts{template <RandomAccessIterator Iter>}@
  @\addedConcepts{void advance(Iter\& i, Iter::difference_type n);}@
\end{itemdecl}

\editorial{Note that we have eliminated the \tcode{Distance} parameter
  in favor of the \tcode{difference_type} of the iterator, which more
  accurately reflects how the iterator can move.}

\begin{itemdescr}
\pnum
\requires\ 
\tcode{n}\
shall be negative only for bidirectional and random access iterators.

\pnum
\effects\ 
Increments (or decrements for negative
\tcode{n})
iterator reference
\tcode{i}\
by
\tcode{n}.
\end{itemdescr}

\index{distance@\tcode{distance}}%
\begin{itemdecl}
template <@\changedConcepts{class InputIterator}{InputIterator Iter}@>
  @\changedConcepts{typename iterator_traits<InputIterator>::difference_type}{Iter::difference_type}@
  distance(@\changedConcepts{InputIterator}{Iter}@ first, @\changedConcepts{InputIterator}{Iter}@ last);
@\addedConcepts{template <RandomAccessIterator Iter>}@
  @\addedConcepts{Iter::difference_type}@
  @\addedConcepts{distance(Iter first, Iter last);}@
\end{itemdecl}

\begin{itemdescr}
\pnum
\effects\ 
Returns the number of increments or decrements needed to get from
\tcode{first}\
to
\tcode{last}.

\pnum
\requires\ 
\tcode{last}\
shall be reachable from
\tcode{first}.
\end{itemdescr}

\index{next@\tcode{next}}%
\begin{itemdecl}
template <@\changedConcepts{class InputIterator}{InputIterator Iter}@>
  @\changedConcepts{InputIterator}{Iter}@ next(@\changedConcepts{InputIterator}{Iter}@ x,
    @\changedConcepts{typename std::iterator_traits<InputIterator>::difference_type}{Iter::difference_type}@ n = 1);
\end{itemdecl}

\begin{itemdescr}
\pnum
\effects Equivalent to \tcode{advance(x, n); return x;}
\end{itemdescr}

\index{prev@\tcode{prev}}%
\begin{itemdecl}
template <@\changedConcepts{class BidirectionalIterator}{BidirectionalIterator Iter}@>
  @\changedConcepts{BidirectionalIterator}{Iter}@ prev(@\changedConcepts{BidirectionalIterator}{Iter}@ x,
    @\changedConcepts{typename std::iterator_traits<BidirectionalIterator>::difference_type}{Iter::difference_type}@ n = 1);
\end{itemdecl}

\begin{itemdescr}
\pnum
\effects Equivalent to \tcode{advance(x, -n); return x;}
\end{itemdescr}

\rSec1[predef.iterators]{Predefined iterators}

\rSec2[reverse.iterators]{Reverse iterators}

\pnum
Bidirectional and random access iterators have corresponding reverse iterator adaptors that iterate through
the data structure in the opposite direction.
They have the same signatures as the corresponding iterators.
The fundamental relation between a reverse iterator and its corresponding iterator
\tcode{i}\
is established by the identity:
\tcode{\&*(reverse_iterator(i)) == \&*(i - 1)}.

\pnum
This mapping is dictated by the fact that while there is always a pointer past the end of an array, there might
not be a valid pointer before the beginning of an array.

\rSec3[reverse.iterator]{Class template \tcode{reverse_iterator}}

\index{reverse_iterator@\tcode{reverse_iterator}}%
\begin{codeblock}
namespace std {
  template <@\changedConcepts{class}{BidirectionalIterator}@ Iter@\removedConcepts{ator}@>
  class reverse_iterator @\removedConcepts{: public}@
        @\removedConcepts{iterator<typename iterator_traits<Iterator>::iterator_category,}@
        @\removedConcepts{typename iterator_traits<Iterator>::value_type,}@
        @\removedConcepts{typename iterator_traits<Iterator>::difference_type,}@
        @\removedConcepts{typename iterator_traits<Iterator>::pointer,}@
        @\removedConcepts{typename iterator_traits<Iterator>::reference>}@ {
  protected:
    Iter@\removedConcepts{ator}@ current;
  public:
    typedef Iter@\removedConcepts{ator}@ iterator_type;
    @\addedConcepts{typedef Iter::value_type value_type;}@
    typedef @\changedConcepts{typename iterator_traits<Iterator>::difference_type}{Iter::difference_type}@ difference_type;
    typedef @\changedConcepts{typename iterator_traits<Iterator>::reference}{Iter::reference}@ reference;
    typedef @\changedConcepts{typename iterator_traits<Iterator>::pointer}{Iter::pointer}@ pointer;

    reverse_iterator();
    explicit reverse_iterator(Iter@\removedConcepts{ator}@ x);
    template <class U> 
      @\addedConcepts{requires HasConstructor<Iter, const U\&>}@ 
      reverse_iterator(const reverse_iterator<U>& u);
    template <class U> 
      @\addedConcepts{requires HasAssign<Iter, const U\&>}@
      reverse_iterator operator=(const reverse_iterator<U>& u);

    Iter@\removedConcepts{ator}@ base() const;      // explicit
    reference operator*() const;
    pointer   operator->() const;

    reverse_iterator& operator++();
    reverse_iterator  operator++(int);
    reverse_iterator& operator--();
    reverse_iterator  operator--(int);

    @\addedConcepts{requires RandomAccessIterator<Iter>}@ reverse_iterator  operator+ (difference_type n) const;
    @\addedConcepts{requires RandomAccessIterator<Iter>}@ reverse_iterator& operator+=(difference_type n);
    @\addedConcepts{requires RandomAccessIterator<Iter>}@ reverse_iterator  operator- (difference_type n) const;
    @\addedConcepts{requires RandomAccessIterator<Iter>}@ reverse_iterator& operator-=(difference_type n);
    @\addedConcepts{requires RandomAccessIterator<Iter>}@ @\unspec@ operator[](difference_type n) const;
  };

  template <@\changedConcepts{class}{BidirectionalIterator}@ Iter@\removedConcepts{ator}@1, @\changedConcepts{class}{BidirectionalIterator}@ Iter@\removedConcepts{ator}@2>
    @\addedConcepts{requires HasEqualTo<Iter1, Iter2>}@
    bool operator==(
      const reverse_iterator<Iter@\removedConcepts{ator}@1>& x,
      const reverse_iterator<Iter@\removedConcepts{ator}@2>& y);
  template <@\changedConcepts{class}{RandomAccessIterator}@ Iter@\removedConcepts{ator}@1, @\changedConcepts{class}{RandomAccessIterator}@ Iter@\removedConcepts{ator}@2>
    @\addedConcepts{requires HasGreater<Iter1, Iter2>}@
    bool operator<(
      const reverse_iterator<Iter@\removedConcepts{ator}@1>& x,
      const reverse_iterator<Iter@\removedConcepts{ator}@2>& y);
  template <@\changedConcepts{class}{BidirectionalIterator}@ Iter@\removedConcepts{ator}@1, @\changedConcepts{class}{BidirectionalIterator}@ Iter@\removedConcepts{ator}@2>
    @\addedConcepts{requires HasNotEqualTo<Iter1, Iter2>}@
    bool operator!=(
      const reverse_iterator<Iter@\removedConcepts{ator}@1>& x,
      const reverse_iterator<Iter@\removedConcepts{ator}@2>& y);
  template <@\changedConcepts{class}{RandomAccessIterator}@ Iter@\removedConcepts{ator}@1, @\changedConcepts{class}{RandomAccessIterator}@ Iter@\removedConcepts{ator}@2>
    @\addedConcepts{requires HasLess<Iter1, Iter2>}@
    bool operator>(
      const reverse_iterator<Iter@\removedConcepts{ator}@1>& x,
      const reverse_iterator<Iter@\removedConcepts{ator}@2>& y);
  template <@\changedConcepts{class}{RandomAccessIterator}@ Iter@\removedConcepts{ator}@1, @\changedConcepts{class}{RandomAccessIterator}@ Iter@\removedConcepts{ator}@2>
    @\addedConcepts{requires HasLessEqual<Iter1, Iter2>}@
    bool operator>=(
      const reverse_iterator<Iter@\removedConcepts{ator}@1>& x,
      const reverse_iterator<Iter@\removedConcepts{ator}@2>& y);
  template <@\changedConcepts{class}{RandomAccessIterator}@ Iter@\removedConcepts{ator}@1, @\changedConcepts{class}{RandomAccessIterator}@ Iter@\removedConcepts{ator}@2>
    @\addedConcepts{requires HasGreaterEqual<Iter1, Iter2>}@
    bool operator<=(
      const reverse_iterator<Iter@\removedConcepts{ator}@1>& x,
      const reverse_iterator<Iter@\removedConcepts{ator}@2>& y);
  template <@\changedConcepts{class}{RandomAccessIterator}@ Iter@\removedConcepts{ator}@1, @\changedConcepts{class}{RandomAccessIterator}@ Iter@\removedConcepts{ator}@2>
    @\addedConcepts{requires HasMinus<Iter2, Iter1>}@
    auto operator-(
      const reverse_iterator<Iter@\removedConcepts{ator}@1>& x,
      const reverse_iterator<Iter@\removedConcepts{ator}@2>& y) -> decltype(y.base() - x.base());
  template <@\changedConcepts{class}{RandomAccessIterator}@ Iterator>
    reverse_iterator<Iter@\removedConcepts{ator}@> operator+(
      @\changedConcepts{typename reverse_iterator<Iterator>::difference_type}{Iter::difference_type}@ n,
      const reverse_iterator<Iter@\removedConcepts{ator}@>& x);

  @\addedConcepts{template<BidirectionalIterator Iter>}@
  @\addedConcepts{concept_map BidirectionalIterator<reverse_iterator<Iter> > \{ \}}@

  @\addedConcepts{template<RandomAccessIterator Iter>}@
  @\addedConcepts{concept_map RandomAccessIterator<reverse_iterator<Iter> > \{ \}}@
}
\end{codeblock}

\rSec3[reverse.iter.requirements]{\tcode{reverse_iterator}\ requirements}

\editorial{Remove [reverse.iter.requirements]}

\pnum
\removedConcepts{The template parameter
\mbox{\tcode{Iterator}}
shall meet all the requirements of a Bidirectional Iterator (\mbox{\ref{bidirectional.iterators}}).}

\pnum
\removedConcepts{Additionally,
\mbox{\tcode{Iterator}}
shall meet the requirements of a Random Access Iterator (\mbox{\ref{random.access.iterators}})
if any of the members
\mbox{\tcode{operator+}}
(\mbox{\ref{reverse.iter.op+}}),
\mbox{\tcode{operator-}}
(\mbox{\ref{reverse.iter.op-}}),
\mbox{\tcode{operator+=}}
(\mbox{\ref{reverse.iter.op+=}}),
\mbox{\tcode{operator-=}}
(\mbox{\ref{reverse.iter.op-=}}),
\mbox{\tcode{operator\,[]}}
(\mbox{\ref{reverse.iter.opindex}}),
or the global operators
\mbox{\tcode{operator<}}
(\mbox{\ref{reverse.iter.op<}}),
\mbox{\tcode{operator>}}
(\mbox{\ref{reverse.iter.op>}}),
\mbox{\tcode{operator\,<=}}
(\mbox{\ref{reverse.iter.op<=}}),
\mbox{\tcode{operator>=}}
(\mbox{\ref{reverse.iter.op>=}}),
\mbox{\tcode{operator-}}
(\mbox{\ref{reverse.iter.opdiff}})
or
\mbox{\tcode{operator+}}
(\mbox{\ref{reverse.iter.opsum}}).
is referenced in a way that requires instantiation (\mbox{\ref{temp.inst}}).}

\rSec3[reverse.iter.ops]{\tcode{reverse_iterator}\ operations}

\rSec4[reverse.iter.cons]{\tcode{reverse_iterator}\ constructor}

\index{reverse_iterator@\tcode{reverse_iterator}!\tcode{reverse_iterator}}%
\begin{itemdecl}
reverse_iterator();
\end{itemdecl}

\begin{itemdescr}
\pnum
\effects\
Default initializes
\tcode{current}.
Iterator operations applied to the resulting iterator have defined behavior
if and only if the corresponding operations are defined on a default
constructed iterator of type
\tcode{Iterator}.
\end{itemdescr}

\begin{itemdecl}
explicit reverse_iterator(Iter@\removedConcepts{ator}@ x);
\end{itemdecl}

\begin{itemdescr}
\pnum
\effects\ 
Initializes
\tcode{current}\
with \farg{x}.
\end{itemdescr}

\begin{itemdecl}
template <class U> 
  @\addedConcepts{requires HasConstructor<Iter, const U\&>}@ 
  reverse_iterator(const reverse_iterator<U> &u);
\end{itemdecl}

\begin{itemdescr}
\pnum
\effects\ 
Initializes
\tcode{current}\
with
\tcode{\farg{u}.current}.
\end{itemdescr}

\rSec4[reverse.iter.op=]{\tcode{reverse_iterator::operator=}}

\index{operator=@\tcode{operator=}!\tcode{reverse_iterator}}%
\begin{itemdecl}
template <class U>
  @\addedConcepts{requires HasAssign<Iter, const U\&>}@
  reverse_iterator&
    operator=(const reverse_iterator<U>& u);
\end{itemdecl}

\begin{itemdescr}
\pnum
\effects
Assigns \tcode{u.base()} to current.

\pnum
\returns
\tcode{*this}.
\end{itemdescr}

\rSec4[reverse.iter.conv]{Conversion}

\index{conversion!reverse_iterator@\tcode{reverse_iterator}}%
\begin{itemdecl}
Iter@\removedConcepts{ator}@ base() const;          // explicit
\end{itemdecl}

\begin{itemdescr}
\pnum
\returns\ 
\tcode{current}.
\end{itemdescr}

\rSec4[reverse.iter.op.star]{\tcode{operator*}}

\index{operator*@\tcode{operator*}!\tcode{reverse_iterator}}%
\begin{itemdecl}
reference operator*() const;
\end{itemdecl}

\begin{itemdescr}
\pnum
\effects\ 
\begin{codeblock}
this->tmp = current;
--this->tmp;
return *this->tmp;
\end{codeblock}

\pnum
\enternote\
This operation must use an auxiliary member variable, rather than a
temporary variable, to avoid returning a reference that persists beyond
the lifetime of its associated iterator.
(See \ref{iterator.requirements}.)
The name of this member variable is shown for exposition only.
\exitnote\
\end{itemdescr}

\rSec4[reverse.iter.opref]{\tcode{operator->}}

\index{operator->@\tcode{operator->}!\tcode{reverse_iterator}}%
\begin{itemdecl}
pointer operator->() const;
\end{itemdecl}

\begin{itemdescr}
\pnum
\returns
\begin{codeblock}
&(operator*());
\end{codeblock}
\end{itemdescr}

\rSec4[reverse.iter.op++]{\tcode{operator++}}

\index{operator++@\tcode{operator++}!\tcode{reverse_iterator}}%
\begin{itemdecl}
reverse_iterator& operator++();
\end{itemdecl}

\begin{itemdescr}
\pnum
\effects\ 
\tcode{\dcr current;}

\pnum
\returns\ 
\tcode{*this}.
\end{itemdescr}

\begin{itemdecl}
reverse_iterator operator++(int);
\end{itemdecl}

\begin{itemdescr}
\pnum
\effects\ 
\begin{codeblock}
reverse_iterator tmp = *this;
--current;
return tmp;
\end{codeblock}
\end{itemdescr}

\rSec4[reverse.iter.op\dcr]{\tcode{operator\dcr}}

\index{operator\dcr@\tcode{operator\dcr}!\tcode{reverse_iterator}}%
\begin{itemdecl}
reverse_iterator& operator--();
\end{itemdecl}

\begin{itemdescr}
\pnum
\effects\ 
\tcode{++current}

\pnum
\returns\ 
\tcode{*this}.
\end{itemdescr}

\begin{itemdecl}
reverse_iterator operator--(int);
\end{itemdecl}

\begin{itemdescr}
\pnum
\effects\ 
\begin{codeblock}
reverse_iterator tmp = *this;
++current;
return tmp;
\end{codeblock}
\end{itemdescr}

\rSec4[reverse.iter.op+]{\tcode{operator+}}

\index{operator+@\tcode{operator+}!\tcode{reverse_iterator}}%
\begin{itemdecl}
@\addedConcepts{requires RandomAccessIterator<Iter>}@
reverse_iterator
operator+(@\removedConcepts{typename reverse_iterator<Iterator>::}@difference_type n) const;
\end{itemdecl}

\begin{itemdescr}
\pnum
\returns\ 
\tcode{reverse_iterator(current-n)}.
\end{itemdescr}

\rSec4[reverse.iter.op+=]{\tcode{operator+=}}

\index{operator+=@\tcode{operator+=}!\tcode{reverse_iterator}}%
\begin{itemdecl}
@\addedConcepts{requires RandomAccessIterator<Iter>}@
reverse_iterator&
operator+=(@\removedConcepts{typename reverse_iterator<Iterator>::}@difference_type n);
\end{itemdecl}

\begin{itemdescr}
\pnum
\effects\ 
\tcode{current -= n;}\

\pnum
\returns\ 
\tcode{*this}.
\end{itemdescr}

\rSec4[reverse.iter.op-]{\tcode{operator-}}

\index{operator-@\tcode{operator-}!\tcode{reverse_iterator}}%
\begin{itemdecl}
@\addedConcepts{requires RandomAccessIterator<Iter>}@
reverse_iterator
operator-(@\removedConcepts{typename reverse_iterator<Iterator>::}@difference_type n) const;
\end{itemdecl}

\begin{itemdescr}
\pnum
\returns\ 
\tcode{reverse_iterator(current+n)}.
\end{itemdescr}

\rSec4[reverse.iter.op-=]{\tcode{operator-=}}

\index{operator-=@\tcode{operator-=}!\tcode{reverse_iterator}}%
\begin{itemdecl}
@\addedConcepts{requires RandomAccessIterator<Iter>}@
reverse_iterator&
operator-=(@\removedConcepts{typename reverse_iterator<Iterator>::}@difference_type n);
\end{itemdecl}

\begin{itemdescr}
\pnum
\effects\ 
\tcode{current += n;}\

\pnum
\returns\ 
\tcode{*this}.
\end{itemdescr}

\rSec4[reverse.iter.opindex]{\tcode{operator[]}}

\index{operator[]@\tcode{operator[]}!reverse_iterator@\tcode{reverse_iterator}}%
\begin{itemdecl}
@\addedConcepts{requires RandomAccessIterator<Iter>}@
@\unspec@ operator[](
    @\removedConcepts{typename reverse_iterator<Iterator>::}@difference_type n) const;
\end{itemdecl}

\begin{itemdescr}
\pnum
\returns\ 
\tcode{current[-n-1]}.
\end{itemdescr}

\rSec4[reverse.iter.op==]{\tcode{operator==}}

\index{operator==@\tcode{operator==}!\tcode{reverse_iterator}}%
\begin{itemdecl}
template <@\changedConcepts{class}{BidirectionalIterator}@ Iter@\removedConcepts{ator}@1, @\changedConcepts{class}{BidirectionalIterator}@ Iter@\removedConcepts{ator}@2>
  @\addedConcepts{requires HasEqualTo<Iter1, Iter2>}@
  bool operator==(
    const reverse_iterator<Iter@\removedConcepts{ator}@1>& x,
    const reverse_iterator<Iter@\removedConcepts{ator}@2>& y);
\end{itemdecl}

\begin{itemdescr}
\pnum
\returns\ 
\tcode{x.current == y.current}.
\end{itemdescr}

\rSec4[reverse.iter.op<]{\tcode{operator<}}

\index{operator<@\tcode{operator<}!\tcode{reverse_iterator}}%
\begin{itemdecl}
template <@\changedConcepts{class}{RandomAccessIterator}@ Iter@\removedConcepts{ator}@1, @\changedConcepts{class}{RandomAccessIterator}@ Iter@\removedConcepts{ator}@2>
  @\addedConcepts{requires HasGreater<Iter1, Iter2>}@
  bool operator<(
    const reverse_iterator<Iter@\removedConcepts{ator}@1>& x,
    const reverse_iterator<Iter@\removedConcepts{ator}@2>& y);
\end{itemdecl}

\begin{itemdescr}
\pnum
\returns\ 
\tcode{x.current > y.current}.
\end{itemdescr}

\rSec4[reverse.iter.op!=]{\tcode{operator!=}}

\index{operator"!=@\tcode{operator"!=}!\tcode{reverse_iterator}}%
\begin{itemdecl}
template <@\changedConcepts{class}{BidirectionalIterator}@ Iter@\removedConcepts{ator}@1, @\changedConcepts{class}{BidirectionalIterator}@ Iter@\removedConcepts{ator}@2>
  @\addedConcepts{requires HasNotEqualTo<Iter1, Iter2>}@
  bool operator!=(
    const reverse_iterator<Iter@\removedConcepts{ator}@1>& x,
    const reverse_iterator<Iter@\removedConcepts{ator}@2>& y);
\end{itemdecl}

\begin{itemdescr}
\pnum
\returns\ 
\tcode{x.current != y.current}.
\end{itemdescr}

\rSec4[reverse.iter.op>]{\tcode{operator>}}

\index{operator>@\tcode{operator>}!\tcode{reverse_iterator}}%
\begin{itemdecl}
template <@\changedConcepts{class}{RandomAccessIterator}@ Iter@\removedConcepts{ator}@1, @\changedConcepts{class}{RandomAccessIterator}@ Iter@\removedConcepts{ator}@2>
  @\addedConcepts{requires HasLess<Iter1, Iter2>}@
  bool operator>(
    const reverse_iterator<Iter@\removedConcepts{ator}@1>& x,
    const reverse_iterator<Iter@\removedConcepts{ator}@2>& y);
\end{itemdecl}

\begin{itemdescr}
\pnum
\returns\ 
\tcode{x.current < y.current}.
\end{itemdescr}

\rSec4[reverse.iter.op>=]{\tcode{operator>=}}

\index{operator>=@\tcode{operator>=}!\tcode{reverse_iterator}}%
\begin{itemdecl}
template <@\changedConcepts{class}{RandomAccessIterator}@ Iter@\removedConcepts{ator}@1, @\changedConcepts{class}{RandomAccessIterator}@ Iter@\removedConcepts{ator}@2>
  @\addedConcepts{requires HasLessEqual<Iter1, Iter2>}@
  bool operator>=(
    const reverse_iterator<Iter@\removedConcepts{ator}@1>& x,
    const reverse_iterator<Iter@\removedConcepts{ator}@2>& y);
\end{itemdecl}

\begin{itemdescr}
\pnum
\returns\ 
\tcode{x.current <= y.current}.
\end{itemdescr}

\rSec4[reverse.iter.op<=]{\tcode{operator<=}}

\index{operator<=@\tcode{operator<=}!\tcode{reverse_iterator}}%
\begin{itemdecl}
template <@\changedConcepts{class}{RandomAccessIterator}@ Iter@\removedConcepts{ator}@1, @\changedConcepts{class}{RandomAccessIterator}@ Iter@\removedConcepts{ator}@2>
  @\addedConcepts{requires HasGreaterEqual<Iter1, Iter2>}@
  bool operator<=(
    const reverse_iterator<Iter@\removedConcepts{ator}@1>& x,
    const reverse_iterator<Iter@\removedConcepts{ator}@2>& y);
\end{itemdecl}

\begin{itemdescr}
\pnum
\returns\ 
\tcode{x.current >= y.current}.
\end{itemdescr}

\rSec4[reverse.iter.opdiff]{\tcode{operator-}}

\index{operator-@\tcode{operator-}!\tcode{reverse_iterator}}%
\begin{itemdecl}
template <@\changedConcepts{class}{RandomAccessIterator}@ Iter@\removedConcepts{ator}@1, @\changedConcepts{class}{RandomAccessIterator}@ Iter@\removedConcepts{ator}@2>
  @\addedConcepts{requires HasMinus<Iter2, Iter1>}@
  auto operator-(
    const reverse_iterator<Iter@\removedConcepts{ator}@1>& x,
    const reverse_iterator<Iter@\removedConcepts{ator}@2>& y) -> decltype(y.base() - x.base());
\end{itemdecl}

\begin{itemdescr}
\pnum
\returns\ 
\tcode{y.current - x.current}.
\end{itemdescr}

\rSec4[reverse.iter.opsum]{\tcode{operator+}}

\index{operator+@\tcode{operator+}!\tcode{reverse_iterator}}%
\begin{itemdecl}
template <@\changedConcepts{class}{RandomAccessIterator}@ Iterator>
  reverse_iterator<Iter@\removedConcepts{ator}@> operator+(
    @\changedConcepts{typename reverse_iterator<Iterator>::difference_type}{Iter::difference_type}@ n,
    const reverse_iterator<Iter@\removedConcepts{ator}@>& x);
\end{itemdecl}

\begin{itemdescr}
\pnum
\returns\ 
\tcode{reverse_iterator<Iter\removedConcepts{ator}> (x.current - n)}.
\end{itemdescr}

\rSec3[reverse.iter.maps]{Concept maps}
\begin{itemdecl}
@\addedConcepts{template<BidirectionalIterator Iter>}@
@\addedConcepts{concept_map BidirectionalIterator<reverse_iterator<Iter> > \{ \}}@
\end{itemdecl}

\begin{itemdescr}
\pnum
\addedConcepts{\reallynote This concept map template states that reverse iterators are themselves bidirectional iterators.}
\end{itemdescr}

\begin{itemdecl}
@\addedConcepts{template<RandomAccessIterator Iter>}@
@\addedConcepts{concept_map RandomAccessIterator<reverse_iterator<Iter> > \{ \}}@
\end{itemdecl}

\begin{itemdescr}
\pnum
\addedConcepts{\reallynote This concept map template states that reverse iterators are themselves random access iterators when the underlying iterator is a random access iterator.}
\end{itemdescr}

\rSec2[insert.iterators]{Insert iterators}

\pnum
To make it possible to deal with insertion in the same way as writing into an array, a special kind of iterator
adaptors, called
\techterm{insert iterators},
are provided in the library.
With regular iterator classes,

\begin{codeblock}
while (first != last) *result++ = *first++;
\end{codeblock}

causes a range \range{first}{last}\
to be copied into a range starting with result.
The same code with
\tcode{result}\
being an insert iterator will insert corresponding elements into the container.
This device allows all of the
copying algorithms in the library to work in the
\techterm{insert mode}\ 
instead of the \techterm{regular overwrite}\ mode.

\pnum
An insert iterator is constructed from a container and possibly one of its iterators pointing to where
insertion takes place if it is neither at the beginning nor at the end of the container.
Insert iterators satisfy the requirements of output iterators.
\tcode{operator*}\
returns the insert iterator itself.
The assignment
\tcode{operator=(const T\& x)}\
is defined on insert iterators to allow writing into them, it inserts
\tcode{x}\
right before where the insert iterator is pointing.
In other words, an insert iterator is like a cursor pointing into the
container where the insertion takes place.
\tcode{back_insert_iterator}\
inserts elements at the end of a container,
\tcode{front_insert_iterator}\
inserts elements at the beginning of a container, and
\tcode{insert_iterator}\
inserts elements where the iterator points to in a container.
\tcode{back_inserter},
\tcode{front_inserter},
and
\tcode{inserter}\
are three
functions making the insert iterators out of a container.

\rSec3[back.insert.iterator]{Class template \tcode{back_insert_iterator}}

\index{back_insert_iterator@\tcode{back_insert_iterator}}%
\begin{codeblock}
namespace std {
  template <@\changedConcepts{class}{BackInsertionContainer}@ Cont@\removedConcepts{ainer}@>
  class back_insert_iterator @\removedConcepts{:}@
    @\removedConcepts{public iterator<output_iterator_tag,void,void,void,void>}@ {
  protected:
    Cont@\removedConcepts{ainer}@* container;

  public:
    typedef Cont@\removedConcepts{ainer}@ container_type;
    @\addedConcepts{typedef void                             value_type;}@
    @\addedConcepts{typedef void                             difference_type;}@
    @\addedConcepts{typedef back_insert_iterator<Cont>\&      reference;}@
    @\addedConcepts{typedef void                             pointer;}@

    explicit back_insert_iterator(Cont@\removedConcepts{ainer}@& x);
    back_insert_iterator<Cont@\removedConcepts{ainer}@>&
      operator=(@\changedConcepts{typename}{const}@ Cont@\removedConcepts{ainer}@::@\changedConcepts{const_reference}{value_type\&}@ value);
    back_insert_iterator<Cont@\removedConcepts{ainer}@>&
      operator=(@\removedConcepts{typename }@Cont@\removedConcepts{ainer}@::value_type&& value);

    back_insert_iterator<Cont@\removedConcepts{ainer}@>& operator*();
    back_insert_iterator<Cont@\removedConcepts{ainer}@>& operator++();
    back_insert_iterator<Cont@\removedConcepts{ainer}@>  operator++(int);
  };

  template <@\changedConcepts{class}{BackInsertionContainer}@ Cont@\removedConcepts{ainer}@>
    back_insert_iterator<Cont@\removedConcepts{ainer}@> back_inserter(Cont@\removedConcepts{ainer}@& x);

  @\addedConcepts{template<BackInsertionContainer Cont>}@
    @\addedConcepts{concept_map Iterator<back_insert_iterator<Cont> > \{ \}}@
}
\end{codeblock}

\rSec3[back.insert.iter.ops]{\tcode{back_insert_iterator}\ operations}

\rSec4[back.insert.iter.cons]{\tcode{back_insert_iterator}\ constructor}

\index{back_insert_iterator@\tcode{back_insert_iterator}!\tcode{back_insert_iterator}}%
\begin{itemdecl}
explicit back_insert_iterator(Cont@\removedConcepts{ainer}@& x);
\end{itemdecl}

\begin{itemdescr}
\pnum
\effects\ 
Initializes
\tcode{container}\
with \tcode{\&\farg{x}}.
\end{itemdescr}

\rSec4[back.insert.iter.op=]{\tcode{back_insert_iterator::operator=}}

\index{operator=@\tcode{operator=}!\tcode{back_insert_iterator}}%
\begin{itemdecl}
back_insert_iterator<Cont@\removedConcepts{ainer}@>&
  operator=(@\changedConcepts{typename}{const}@ Cont@\removedConcepts{ainer}@::@\changedConcepts{const_reference}{value_type\&}@ value);
\end{itemdecl}

\begin{itemdescr}
\pnum
\effects\ 
\tcode{\removedConcepts{container->}push_back(\addedConcepts{*container, }value);}\

\pnum
\returns\ 
\tcode{*this}.
\end{itemdescr}

\index{operator=@\tcode{operator=}!\tcode{back_insert_iterator}}%
\begin{itemdecl}
back_insert_iterator<Cont@\removedConcepts{ainer}@>&
  operator=(@\removedConcepts{typename }@Cont@\removedConcepts{ainer}@::value_type&& value);
\end{itemdecl}

\begin{itemdescr}
\pnum
\effects
\tcode{\removedConcepts{container->}push_back(\addedConcepts{*container, }std::move(value));}

\pnum
\returns
\tcode{*this}.
\end{itemdescr}

\rSec4[back.insert.iter.op*]{\tcode{back_insert_iterator::operator*}}

\index{operator*@\tcode{operator*}!\tcode{back_insert_iterator}}%
\begin{itemdecl}
back_insert_iterator<Cont@\removedConcepts{ainer}@>& operator*();
\end{itemdecl}

\begin{itemdescr}
\pnum
\returns\ 
\tcode{*this}.
\end{itemdescr}

\rSec4[back.insert.iter.op++]{\tcode{back_insert_iterator::operator++}}

\index{operator++@\tcode{operator++}!\tcode{back_insert_iterator}}%
\begin{itemdecl}
back_insert_iterator<Cont@\removedConcepts{ainer}@>& operator++();
back_insert_iterator<Cont@\removedConcepts{ainer}@>  operator++(int);
\end{itemdecl}

\begin{itemdescr}
\pnum
\returns\ 
\tcode{*this}.
\end{itemdescr}

\rSec4[back.inserter]{\ \tcode{back_inserter}}

\index{back_inserter@\tcode{back_inserter}}%
\begin{itemdecl}
template <@\changedConcepts{class}{BackInsertionContainer}@ Cont@\removedConcepts{ainer}@>
  back_insert_iterator<Cont@\removedConcepts{ainer}@> back_inserter(Cont@\removedConcepts{ainer}@& x);
\end{itemdecl}

\begin{itemdescr}
\pnum
\returns\ 
\tcode{back_insert_iterator<Cont@\removedConcepts{ainer}@>(x)}.
\end{itemdescr}

\rSec4[back.insert.iter.maps]{Concept maps}
\begin{itemdecl}
@\addedConcepts{template<BackInsertionContainer Cont>}@
  @\addedConcepts{concept_map Iterator<back_insert_iterator<Cont> > \{ \}}@
\end{itemdecl}

\begin{itemdescr}
\pnum
\addedConcepts{\reallynote Declares that \mbox{\tcode{back_insert_iterator}} is an iterator.}
\end{itemdescr}

\rSec3[front.insert.iterator]{Class template \tcode{front_insert_iterator}}

\index{front_insert_iterator@\tcode{front_insert_iterator}}%
\begin{codeblock}
namespace std {
  template <@\changedConcepts{class}{FrontInsertionContainer}@ Cont@\removedConcepts{ainer}@>
  class front_insert_iterator @\removedConcepts{:}@
    @\removedConcepts{public iterator<output_iterator_tag,void,void,void,void>}@ {
  protected:
    Cont@\removedConcepts{ainer}@* container;

  public:
    typedef Cont@\removedConcepts{ainer}@ container_type;
    @\addedConcepts{typedef void                              value_type;}@
    @\addedConcepts{typedef void                              difference_type;}@
    @\addedConcepts{typedef front_insert_iterator<Cont>\&      reference;}@
    @\addedConcepts{typedef void                              pointer;}@

    explicit front_insert_iterator(Cont@\removedConcepts{ainer}@& x);
    front_insert_iterator<Cont@\removedConcepts{ainer}@>&
      operator=(@\changedConcepts{typename}{const}@ Cont@\removedConcepts{ainer}@::@\changedConcepts{const_reference}{value_type\&}@ value);
    front_insert_iterator<Cont@\removedConcepts{ainer}@>&
      operator=(@\removedConcepts{typename}@ Cont@\removedConcepts{ainer}@::value_type&& value);

    front_insert_iterator<Cont@\removedConcepts{ainer}@>& operator*();
    front_insert_iterator<Cont@\removedConcepts{ainer}@>& operator++();
    front_insert_iterator<Cont@\removedConcepts{ainer}@>  operator++(int);
  };

  template <@\changedConcepts{class}{FrontInsertionContainer}@ Cont@\removedConcepts{ainer}@>
    front_insert_iterator<Cont@\removedConcepts{ainer}@> front_inserter(Cont@\removedConcepts{ainer}@& x);

  @\addedConcepts{template<FrontInsertionContainer Cont>}@
    @\addedConcepts{concept_map Iterator<front_insert_iterator<Cont> > \{ \}}@
}
\end{codeblock}

\rSec3[front.insert.iter.ops]{\tcode{front_insert_iterator}\ operations}

\rSec4[front.insert.iter.cons]{\tcode{front_insert_iterator}\ constructor}

\index{front_insert_iterator@\tcode{front_insert_iterator}!\tcode{front_insert_iterator}}%
\begin{itemdecl}
explicit front_insert_iterator(Cont@\removedConcepts{ainer}@& x);
\end{itemdecl}

\begin{itemdescr}
\pnum
\effects\ 
Initializes
\tcode{container}
with \tcode{\&}\farg{x}.
\end{itemdescr}

\rSec4[front.insert.iter.op=]{\tcode{front_insert_iterator::operator=}}

\index{operator=@\tcode{operator=}!\tcode{front_insert_iterator}}%
\begin{itemdecl}
front_insert_iterator<Cont@\removedConcepts{ainer}@>&
  operator=(@\changedConcepts{typename}{const}@ Cont@\removedConcepts{ainer}@::@\changedConcepts{const_reference}{value_type\&}@ value);
\end{itemdecl}

\begin{itemdescr}
\pnum
\effects\ 
\tcode{\removedConcepts{container->}push_front(\addedConcepts{*container, }value);}\

\pnum
\returns\ 
\tcode{*this}.
\end{itemdescr}

\index{operator=@\tcode{operator=}!\tcode{front_insert_iterator}}%
\begin{itemdecl}
front_insert_iterator<Cont@\removedConcepts{ainer}@>&
  operator=(@\removedConcepts{typename }@Cont@\removedConcepts{ainer}@::value_type&& value);
\end{itemdecl}

\begin{itemdescr}
\pnum
\effects
\tcode{\removedConcepts{container->}push_front(\addedConcepts{*container, }std::move(value));}

\pnum
\returns
\tcode{*this}.
\end{itemdescr}

\rSec4[front.insert.iter.op*]{\tcode{front_insert_iterator::operator*}}

\index{operator*@\tcode{operator*}!\tcode{front_insert_iterator}}%
\begin{itemdecl}
front_insert_iterator<Cont@\removedConcepts{ainer}@>& operator*();
\end{itemdecl}

\begin{itemdescr}
\pnum
\returns\ 
\tcode{*this}.
\end{itemdescr}

\rSec4[front.insert.iter.op++]{\tcode{front_insert_iterator::operator++}}

\index{operator++@\tcode{operator++}!\tcode{front_insert_iterator}}%
\begin{itemdecl}
front_insert_iterator<Cont@\removedConcepts{ainer}@>& operator++();
front_insert_iterator<Cont@\removedConcepts{ainer}@>  operator++(int);
\end{itemdecl}

\begin{itemdescr}
\pnum
\returns\ 
\tcode{*this}.
\end{itemdescr}

\rSec4[front.inserter]{\tcode{front_inserter}}

\index{front_inserter@\tcode{front_inserter}}%
\begin{itemdecl}
template <@\changedConcepts{class}{FrontInsertionContainer}@ Cont@\removedConcepts{ainer}@>
  front_insert_iterator<Cont@\removedConcepts{ainer}@> front_inserter(Cont@\removedConcepts{ainer}@& x);
\end{itemdecl}

\begin{itemdescr}
\pnum
\returns\ 
\tcode{front_insert_iterator<Cont@\removedConcepts{ainer}@>(x)}.
\end{itemdescr}

\rSec4[front.insert.iter.maps]{Concept maps}
\begin{itemdecl}
@\addedConcepts{template<FrontInsertionContainer Cont>}@
  @\addedConcepts{concept_map Iterator<front_insert_iterator<Cont> > \{ \}}@
\end{itemdecl}

\begin{itemdescr}
\pnum
\addedConcepts{\reallynote Declares that \mbox{\tcode{front_insert_iterator}} is an iterator.}
\end{itemdescr}

\rSec3[insert.iterator]{Class template \tcode{insert_iterator}}

\index{insert_iterator@\tcode{insert_iterator}}%
\begin{codeblock}
namespace std {
  template <@\changedConcepts{class}{InsertionContainer}@ Cont@\removedConcepts{ainer}@>
  class insert_iterator @\removedConcepts{:}@
    @\removedConcepts{public iterator<output_iterator_tag,void,void,void,void>}@ {
  protected:
    Cont@\removedConcepts{ainer}@* container;
    @\removedConcepts{typename }@Cont@\removedConcepts{ainer}@::iterator iter;

  public:
    typedef Cont@\removedConcepts{ainer}@ container_type;
    @\addedConcepts{typedef void                        value_type;}@
    @\addedConcepts{typedef void                        difference_type;}@
    @\addedConcepts{typedef insert_iterator<Cont>\&      reference;}@
    @\addedConcepts{typedef void                        pointer;}@

    insert_iterator(Cont@\removedConcepts{ainer}@& x, @\removedConcepts{typename }@Cont@\removedConcepts{ainer}@::iterator i);
    insert_iterator<Cont@\removedConcepts{ainer}@>&
      operator=(@\changedConcepts{typename}{const}@ Cont@\removedConcepts{ainer}@::@\changedConcepts{const_reference}{value_type\&}@ value);
    insert_iterator<Cont@\removedConcepts{ainer}@>&
      operator=(@\removedConcepts{typename }@Cont@\removedConcepts{ainer}@::value_type&& value);

    insert_iterator<Cont@\removedConcepts{ainer}@>& operator*();
    insert_iterator<Cont@\removedConcepts{ainer}@>& operator++();
    insert_iterator<Cont@\removedConcepts{ainer}@>& operator++(int);
  };

  template <@\changedConcepts{class}{InsertionContainer}@ Cont@\removedConcepts{ainer}@>
    insert_iterator<Cont@\removedConcepts{ainer}@> inserter(Cont@\removedConcepts{ainer}@& x, Cont@\removedConcepts{ainer}@::iterator i);

  @\addedConcepts{template<InsertionContainer Cont>}@
    @\addedConcepts{concept_map Iterator<insert_iterator<Cont> > \{ \}}@
}
\end{codeblock}

\rSec3[insert.iter.ops]{\tcode{insert_iterator}\ operations}

\rSec4[insert.iter.cons]{\tcode{insert_iterator}\ constructor}

\index{insert_iterator@\tcode{insert_iterator}!\tcode{insert_iterator}}%
\begin{itemdecl}
insert_iterator(Cont@\removedConcepts{ainer}@& x, @\removedConcepts{typename}@ Cont@\removedConcepts{ainer}@::iterator i);
\end{itemdecl}

\begin{itemdescr}
\pnum
\effects\ 
Initializes
\tcode{container}\
with \farg{\&x} and
\tcode{iter}\
with \farg{i}.
\end{itemdescr}

\rSec4[insert.iter.op=]{\tcode{insert_iterator::operator=}}

\index{operator=@\tcode{operator=}!\tcode{insert_iterator}}%
\begin{itemdecl}
insert_iterator<Cont@\removedConcepts{ainer}@>&
  operator=(@\changedConcepts{typename}{const}@ Cont@\removedConcepts{ainer}@::@\changedConcepts{const_reference}{value_type\&}@ value);
\end{itemdecl}

\begin{itemdescr}
\pnum
\effects\ 
\begin{codeblock}
iter = @\removedConcepts{container->}@insert(@\addedConcepts{*container, }@iter, value);
++iter;
\end{codeblock}

\pnum
\returns\ 
\tcode{*this}.
\end{itemdescr}

\index{operator=@\tcode{operator=}!\tcode{insert_iterator}}%
\begin{itemdecl}
insert_iterator<Cont@\removedConcepts{ainer}@>&
  operator=(Cont@\removedConcepts{ainer}@::value_type&& value);
\end{itemdecl}

\begin{itemdescr}
\pnum
\effects
\begin{codeblock}
iter = @\removedConcepts{container->}@insert(@\addedConcepts{*container, }@iter, std::move(value));
++iter;
\end{codeblock}

\pnum
\returns
\tcode{*this}.
\end{itemdescr}

\rSec4[insert.iter.op*]{\tcode{insert_iterator::operator*}}

\index{operator*@\tcode{operator*}!\tcode{insert_iterator}}%
\begin{itemdecl}
insert_iterator<Cont@\removedConcepts{ainer}@>& operator*();
\end{itemdecl}

\begin{itemdescr}
\pnum
\returns\ 
\tcode{*this}.
\end{itemdescr}

\rSec4[insert.iter.op++]{\tcode{insert_iterator::operator++}}

\index{operator++@\tcode{operator++}!\tcode{insert_iterator}}%
\begin{itemdecl}
insert_iterator<Cont@\removedConcepts{ainer}@>& operator++();
insert_iterator<Cont@\removedConcepts{ainer}@>& operator++(int);
\end{itemdecl}

\begin{itemdescr}
\pnum
\returns\ 
\tcode{*this}.
\end{itemdescr}

\rSec4[inserter]{\tcode{inserter}}

\index{inserter@\tcode{inserter}}%
\begin{itemdecl}
template <@\changedConcepts{class}{InsertionContainer}@ Cont@\removedConcepts{ainer}@>
  insert_iterator<Cont@\removedConcepts{ainer}@> inserter(Cont@\removedConcepts{ainer}@& x, @\removedConcepts{typename }@Cont@\removedConcepts{ainer}@::iterator i);
\end{itemdecl}

\begin{itemdescr}
\pnum
\returns\ 
\tcode{insert_iterator<Cont@\removedConcepts{ainer}@>(x, i)}.
\end{itemdescr}

\rSec4[insert.iter.maps]{Concept maps}
\begin{itemdecl}
@\addedConcepts{template<InsertionContainer Cont>}@
  @\addedConcepts{concept_map Iterator<insert_iterator<Cont> > \{ \}}@
\end{itemdecl}

\begin{itemdescr}
\pnum
\addedConcepts{\reallynote Declares that \mbox{\tcode{insert_iterator}} is an iterator.}
\end{itemdescr}

\rSec2[move.iterators]{Move iterators}

\pnum
Class template \tcode{move_iterator} is an iterator adaptor
with the same behavior as the underlying iterator except that its
dereference operator implicitly converts the value returned by the
underlying iterator's dereference operator to an rvalue reference.
Some generic algorithms can be called with move iterators to replace
copying with moving.

\pnum
\enterexample

\begin{codeblock}
set<string> s;
// populate the set \tcode{s}
vector<string> v1(s.begin(), s.end());          // copies strings into \tcode{v1}
vector<string> v2(make_move_iterator(s.begin()),
                  make_move_iterator(s.end())); // moves strings into \tcode{v2}
\end{codeblock}

\exitexample

\rSec3[move.iterator]{Class template \tcode{move_iterator}}

\index{move_iterator@\tcode{move_iterator}}%
\begin{codeblock}
namespace std {
  template <@\changedConcepts{class}{InputIterator}@ Iter@\removedConcepts{ator}@>
  class move_iterator {
  public:
    typedef Iter@\removedConcepts{ator}@                                              iterator_type;
    typedef @\changedConcepts{typename iterator_traits<Iterator>}{Iter}@::difference_type   difference_type;
    typedef Iterator                                              pointer;
    typedef @\changedConcepts{typename iterator_traits<Iterator>}{Iter}@::value_type        value_type;
    @\removedConcepts{typedef typename iterator_traits<Iterator>::iterator_category iterator_category;}@
    typedef value_type&&                                          reference;

    move_iterator();
    explicit move_iterator(Iter@\removedConcepts{ator}@ i);
    template <class U> 
      @\addedConcepts{requires HasConstructor<Iter, const U\&>}@ 
      move_iterator(const move_iterator<U>& u);
    template <class U> 
      @\addedConcepts{requires HasAssign<Iter, const U\&>}@ 
      move_iterator& operator=(const move_iterator<U>& u);

    iterator_type base() const;
    reference operator*() const;
    pointer operator->() const;

    move_iterator& operator++();
    move_iterator operator++(int);
    @\addedConcepts{requires BidirectionalIterator<Iter>}@ move_iterator& operator--();
    @\addedConcepts{requires BidirectionalIterator<Iter>}@ move_iterator operator--(int);

    @\addedConcepts{requires RandomAccessIterator<Iter>}@ move_iterator operator+(difference_type n) const;
    @\addedConcepts{requires RandomAccessIterator<Iter>}@ move_iterator& operator+=(difference_type n);
    @\addedConcepts{requires RandomAccessIterator<Iter>}@ move_iterator operator-(difference_type n) const;
    @\addedConcepts{requires RandomAccessIterator<Iter>}@ move_iterator& operator-=(difference_type n);
    @\addedConcepts{requires RandomAccessIterator<Iter>}@
      @\unspec@ operator[](difference_type n) const;

  private:
    Iter@\removedConcepts{ator}@ current;   // \expos
  };

  template <@\changedConcepts{class}{InputIterator}@ Iter@\removedConcepts{ator}@1, @\changedConcepts{class}{InputIterator}@ Iter@\removedConcepts{ator}@2>
    @\addedConcepts{requires HasEqualTo<Iter1, Iter2>}@
    bool operator==(
      const move_iterator<Iter@\removedConcepts{ator}@1>& x, const move_iterator<Iter@\removedConcepts{ator}@2>& y);
  template <@\changedConcepts{class}{InputIterator}@ Iter@\removedConcepts{ator}@1, @\changedConcepts{class}{InputIterator}@ Iter@\removedConcepts{ator}@2>
    @\addedConcepts{requires HasEqualTo<Iter1, Iter2>}@
    bool operator!=(
      const move_iterator<Iter@\removedConcepts{ator}@1>& x, const move_iterator<Iter@\removedConcepts{ator}@2>& y);
  template <@\changedConcepts{class}{RandomAccessIterator}@ Iter@\removedConcepts{ator}@1, @\changedConcepts{class}{RandomAccessIterator}@ Iter@\removedConcepts{ator}@2>
    @\addedConcepts{requires HasLess<Iter1, Iter2>}@
    bool operator<(
      const move_iterator<Iter@\removedConcepts{ator}@1>& x, const move_iterator<Iter@\removedConcepts{ator}@2>& y);
  template <@\changedConcepts{class}{RandomAccessIterator}@ Iter@\removedConcepts{ator}@1, @\changedConcepts{class}{RandomAccessIterator}@ Iter@\removedConcepts{ator}@2>
    @\addedConcepts{requires HasLess<Iter2, Iter1>}@
    bool operator<=(
      const move_iterator<Iter@\removedConcepts{ator}@1>& x, const move_iterator<Iter@\removedConcepts{ator}@2>& y);
  template <@\changedConcepts{class}{RandomAccessIterator}@ Iter@\removedConcepts{ator}@1, @\changedConcepts{class}{RandomAccessIterator}@ Iter@\removedConcepts{ator}@2>
    @\addedConcepts{requires HasLess<Iter2, Iter1>}@
    bool operator>(
      const move_iterator<Iter@\removedConcepts{ator}@1>& x, const move_iterator<Iter@\removedConcepts{ator}@2>& y);
  template <@\changedConcepts{class}{RandomAccessIterator}@ Iter@\removedConcepts{ator}@1, @\changedConcepts{class}{RandomAccessIterator}@ Iter@\removedConcepts{ator}@2>
    @\addedConcepts{requires HasLess<Iter1, Iter2>}@
    bool operator>=(
      const move_iterator<Iter@\removedConcepts{ator}@1>& x, const move_iterator<Iter@\removedConcepts{ator}@2>& y);

  template <@\changedConcepts{class}{RandomAccessIterator}@ Iter@\removedConcepts{ator}@1, @\changedConcepts{class}{RandomAccessIterator}@ Iter@\removedConcepts{ator}@2>
    @\addedConcepts{requires HasMinus<Iter1, Iter2>}@
    auto operator-(
      const move_iterator<Iter@\removedConcepts{ator}@1>& x, 
      const move_iterator<Iter@\removedConcepts{ator}@2>& y) -> decltype(x.base() - y.base());
  template <@\changedConcepts{class}{RandomAccessIterator}@ Iter@\removedConcepts{ator}@>
    move_iterator<Iter@\removedConcepts{ator}@> operator+(
      @\changedConcepts{typename move_iterator<Iterator>}{Iter}@::difference_type n, const move_iterator<Iter@\removedConcepts{ator}@>& x);
  template <@\changedConcepts{class}{InputIterator}@ Iter@\removedConcepts{ator}@>
    move_iterator<Iter@\removedConcepts{ator}@> make_move_iterator(const Iter@\removedConcepts{ator}@& i);

  @\addedConcepts{template<InputIterator Iter>}@
    @\addedConcepts{concept_map InputIterator<move_iterator<Iter> > \{ \}}@
  @\addedConcepts{template<ForwardIterator Iter>}@
    @\addedConcepts{concept_map ForwardIterator<move_iterator<Iter> > \{ \}}@
  @\addedConcepts{template<BidirectionalIterator Iter>}@
    @\addedConcepts{concept_map BidirectionalIterator<move_iterator<Iter> > \{ \}}@
  @\addedConcepts{template<RandomAccessIterator Iter>}@
    @\addedConcepts{concept_map RandomAccessIterator<move_iterator<Iter> > \{ \}}@
}
\end{codeblock}

\rSec3[move.iter.requirements]{\tcode{move_iterator}\ requirements}

\editorial{Remove [move.iter.requirements]}

\pnum
\removedConcepts{The template parameter \mbox{\tcode{Iterator}} shall meet
the requirements for an Input Iterator~(\mbox{\ref{input.iterators}}).
Additionally, if any of the bidirectional or random access traversal
functions are instantiated, the template parameter shall meet the
requirements for a Bidirectional Iterator~(\mbox{\ref{bidirectional.iterators}})
or a Random Access Iterator~(\mbox{\ref{random.access.iterators}}), respectively.}

\rSec3[move.iter.ops]{\tcode{move_iterator}\ operations}

\rSec4[move.iter.op.const]{\tcode{move_iterator}\ constructors}

\index{move_iterator@\tcode{move_iterator}!\tcode{move_iterator}}%
\begin{itemdecl}
move_iterator();
\end{itemdecl}

\begin{itemdescr}
\pnum
\effects Constructs a \tcode{move_iterator}, default
initializing \tcode{current}.
\end{itemdescr}

\begin{itemdecl}
explicit move_iterator(Iter@\removedConcepts{ator}@ i);
\end{itemdecl}

\begin{itemdescr}
\pnum
\effects Constructs a \tcode{move_iterator}, intializing
\tcode{current} with \tcode{i}.
\end{itemdescr}

\begin{itemdecl}
template <class U> 
  @\addedConcepts{requires HasConstructor<Iter, const U\&>}@ 
  move_iterator(const move_iterator<U>& u);
\end{itemdecl}

\begin{itemdescr}
\pnum
\effects Constructs a \tcode{move_iterator}, initializing
\tcode{current} with \tcode{u.base()}.

\pnum
\removedConcepts{\mbox{\requires} \mbox{\tcode{U}} shall be convertible to
\mbox{\tcode{Iterator}}.}
\end{itemdescr}

\rSec4[move.iter.op=]{\tcode{move_iterator::operator=}}

\begin{itemdecl}
template <class U> 
  @\addedConcepts{requires HasAssign<Iter, const U\&>}@ 
  move_iterator& operator=(const move_iterator<U>& u);
\end{itemdecl}

\begin{itemdescr}
\pnum
\effects Assigns \tcode{u.base()} to
\tcode{current}.

\pnum
\removedConcepts{\mbox{\requires} \mbox{\tcode{U}} shall be convertible to
\mbox{\tcode{Iterator}}.}
\end{itemdescr}

\rSec4[move.iter.op.conv]{\tcode{move_iterator}\ conversion}

\begin{itemdecl}
Iter@\removedConcepts{ator}@ base() const;
\end{itemdecl}

\begin{itemdescr}
\pnum
\returns \tcode{current}.
\end{itemdescr}

\rSec4[move.iter.op.star]{\tcode{move_iterator::operator*}}

\begin{itemdecl}
reference operator*() const;
\end{itemdecl}

\begin{itemdescr}
\pnum
\returns \tcode{*current}, implicitly converted
to an rvalue reference.
\end{itemdescr}

\rSec4[move.iter.op.ref]{\tcode{move_iterator::operator->}}

\begin{itemdecl}
pointer operator->() const;
\end{itemdecl}

\begin{itemdescr}
\pnum
\returns \tcode{current}.
\end{itemdescr}

\rSec4[move.iter.op.incr]{\tcode{move_iterator::operator++}}

\begin{itemdecl}
move_iterator& operator++();
\end{itemdecl}

\begin{itemdescr}
\pnum
\effects \tcode{++current}.

\pnum
\returns \tcode{*this}.
\end{itemdescr}

\begin{itemdecl}
move_iterator& operator++(int);
\end{itemdecl}

\begin{itemdescr}
\pnum
\effects
\begin{codeblock}
move_iterator tmp = *this;
++current;
return tmp;
\end{codeblock}
\end{itemdescr}

\rSec4[move.iter.op.decr]{\tcode{move_iterator::operator-{-}}}

\begin{itemdecl}
@\addedConcepts{requires BidirectionalIterator<Iter>}@ move_iterator& operator--();
\end{itemdecl}

\begin{itemdescr}
\pnum
\effects \tcode{\dcr{}current}.

\pnum
\returns \tcode{*this}.
\end{itemdescr}

\begin{itemdecl}
@\addedConcepts{requires BidirectionalIterator<Iter>}@ move_iterator& operator--(int);
\end{itemdecl}

\begin{itemdescr}
\pnum
\effects
\begin{codeblock}
move_iterator tmp = *this;
--current;
return tmp;
\end{codeblock}
\end{itemdescr}

\rSec4[move.iter.op.+]{\tcode{move_iterator::operator+}}

\begin{itemdecl}
@\addedConcepts{requires RandomAccessIterator<Iter>}@ move_iterator operator+(difference_type n) const;
\end{itemdecl}

\begin{itemdescr}
\pnum
\returns \tcode{move_iterator(current + n)}.
\end{itemdescr}

\rSec4[move.iter.op.+=]{\tcode{move_iterator::operator+=}}

\begin{itemdecl}
@\addedConcepts{requires RandomAccessIterator<Iter>}@ move_iterator& operator+=(difference_type n);
\end{itemdecl}

\begin{itemdescr}
\pnum
\effects \tcode{current += n}.

\pnum
\returns \tcode{*this}.
\end{itemdescr}

\rSec4[move.iter.op.-]{\tcode{move_iterator::operator-}}

\begin{itemdecl}
@\addedConcepts{requires RandomAccessIterator<Iter>}@ move_iterator operator-(difference_type n) const;
\end{itemdecl}

\begin{itemdescr}
\pnum
\returns \tcode{move_iterator(current - n)}.
\end{itemdescr}

\rSec4[move.iter.op.-=]{\tcode{move_iterator::operator-=}}

\begin{itemdecl}
@\addedConcepts{requires RandomAccessIterator<Iter>}@ move_iterator& operator-=(difference_type n);
\end{itemdecl}

\begin{itemdescr}
\pnum
\effects \tcode{current -= n}.

\pnum
\returns \tcode{*this}.
\end{itemdescr}

\rSec4[move.iter.op.index]{\tcode{move_iterator::operator[]}}

\begin{itemdecl}
@\addedConcepts{requires RandomAccessIterator<Iter>}@ 
  @\unspec@ operator[](difference_type n) const;
\end{itemdecl}

\begin{itemdescr}
\pnum
\returns \tcode{current[n]}, implicitly converted to
an rvalue reference.
\end{itemdescr}

\rSec4[move.iter.op.comp]{\tcode{move_iterator}\ comparisons}

\begin{itemdecl}
template <@\changedConcepts{class}{InputIterator}@ Iter@\removedConcepts{ator}@1, @\changedConcepts{class}{InputIterator}@ Iter@\removedConcepts{ator}@2>
  @\addedConcepts{requires HasEqualTo<Iter1, Iter2>}@
  bool operator==(const move_iterator<Iter@\removedConcepts{ator}@1>& x, const move_iterator<Iter@\removedConcepts{ator}@2>& y);
\end{itemdecl}

\begin{itemdescr}
\pnum
\returns \tcode{x.base() == y.base()}.
\end{itemdescr}

\begin{itemdecl}
template <@\changedConcepts{class}{InputIterator}@ Iter@\removedConcepts{ator}@1, @\changedConcepts{class}{InputIterator}@ Iter@\removedConcepts{ator}@2>
  @\addedConcepts{requires HasEqualTo<Iter1, Iter2>}@
  bool operator!=(const move_iterator<Iter@\removedConcepts{ator}@1>& x, const move_iterator<Iter@\removedConcepts{ator}@2>& y);
\end{itemdecl}

\begin{itemdescr}
\pnum
\returns \tcode{!(x == y)}.
\end{itemdescr}

\begin{itemdecl}
template <@\changedConcepts{class}{RandomAccessIterator}@ Iter@\removedConcepts{ator}@1, @\changedConcepts{class}{RandomAccessIterator}@ Iter@\removedConcepts{ator}@2>
  @\addedConcepts{requires HasLess<Iter1, Iter2>}@
  bool operator<(const move_iterator<Iter@\removedConcepts{ator}@1>& x, const move_iterator<Iter@\removedConcepts{ator}@2>& y);
\end{itemdecl}

\begin{itemdescr}
\pnum
\returns \tcode{x.base() < y.base()}.
\end{itemdescr}

\begin{itemdecl}
template <@\changedConcepts{class}{RandomAccessIterator}@ Iter@\removedConcepts{ator}@1, @\changedConcepts{class}{RandomAccessIterator}@ Iter@\removedConcepts{ator}@2>
  @\addedConcepts{requires HasLess<Iter2, Iter1>}@
  bool operator<=(const move_iterator<Iter@\removedConcepts{ator}@1>& x, const move_iterator<Iter@\removedConcepts{ator}@2>& y);
\end{itemdecl}

\begin{itemdescr}
\pnum
\returns \tcode{!(y < x)}.
\end{itemdescr}

\begin{itemdecl}
template <@\changedConcepts{class}{RandomAccessIterator}@ Iter@\removedConcepts{ator}@1, @\changedConcepts{class}{RandomAccessIterator}@ Iter@\removedConcepts{ator}@2>
  @\addedConcepts{requires HasLess<Iter2, Iter1>}@
  bool operator>(const move_iterator<Iter@\removedConcepts{ator}@1>& x, const move_iterator<Iter@\removedConcepts{ator}@2>& y);
\end{itemdecl}

\begin{itemdescr}
\pnum
\returns \tcode{y < x}.
\end{itemdescr}

\begin{itemdecl}
template <@\changedConcepts{class}{RandomAccessIterator}@ Iter@\removedConcepts{ator}@1, @\changedConcepts{class}{RandomAccessIterator}@ Iter@\removedConcepts{ator}@2>
  @\addedConcepts{requires HasLess<Iter1, Iter2>}@
  bool operator>=(const move_iterator<Iter@\removedConcepts{ator}@1>& x, const move_iterator<Iter@\removedConcepts{ator}@2>& y);
\end{itemdecl}

\begin{itemdescr}
\pnum
\returns \tcode{!(x < y)}.
\end{itemdescr}

\rSec4[move.iter.nonmember]{\tcode{move_iterator}\ non-member functions}

\begin{itemdecl}
template <@\changedConcepts{class}{RandomAccessIterator}@ Iter@\removedConcepts{ator}@1, @\changedConcepts{class}{RandomAccessIterator}@ Iter@\removedConcepts{ator}@2>
  @\addedConcepts{requires HasMinus<Iter1, Iter2>}@
  auto operator-(
    const move_iterator<Iter@\removedConcepts{ator}@1>& x, 
    const move_iterator<Iter@\removedConcepts{ator}@2>& y) -> decltype(x.base() - y.base());
\end{itemdecl}

\begin{itemdescr}
\pnum
\returns \tcode{x.base() - y.base()}.
\end{itemdescr}

\begin{itemdecl}
template <@\changedConcepts{class}{RandomAccessIterator}@ Iter@\removedConcepts{ator}@>
  move_iterator<Iter@\removedConcepts{ator}@> operator+(
    @\changedConcepts{typename move_iterator<Iterator>}{Iter}@::difference_type n, const move_iterator<Iter@\removedConcepts{ator}@>& x);
\end{itemdecl}

\begin{itemdescr}
\pnum
\returns \tcode{x + n}.
\end{itemdescr}

\begin{itemdecl}
template <@\changedConcepts{class}{InputIterator}@ Iter@\removedConcepts{ator}@>
move_iterator<Iter@\removedConcepts{ator}@> make_move_iterator(const Iter@\removedConcepts{ator}@& i);
\end{itemdecl}

\begin{itemdescr}
\pnum
\returns \tcode{move_iterator<Iter\removedConcepts{ator}>(i)}.
\end{itemdescr}

\rSec4[move.iter.maps]{Concept maps}
\begin{itemdecl}
@\addedConcepts{template<InputIterator Iter>}@
  @\addedConcepts{concept_map InputIterator<move_iterator<Iter> > \{ \}}@
\end{itemdecl}

\begin{itemdescr}
\pnum
\addedConcepts{\reallynote Declares that a \mbox{\tcode{move_iterator}} is an input iterator.}
\end{itemdescr}

\begin{itemdecl}
@\addedConcepts{template<ForwardIterator Iter>}@
  @\addedConcepts{concept_map ForwardIterator<move_iterator<Iter> > \{ \}}@
\end{itemdecl}

\begin{itemdescr}
\pnum
\addedConcepts{\reallynote Declares that a \mbox{\tcode{move_iterator}} is a forward iterator if its underlying iterator is a forward iterator.}
\end{itemdescr}

\begin{itemdecl}
@\addedConcepts{template<BidirectionalIterator Iter>}@
  @\addedConcepts{concept_map BidirectionalIterator<move_iterator<Iter> > \{ \}}@
\end{itemdecl}

\begin{itemdescr}
\pnum
\addedConcepts{\reallynote Declares that a \mbox{\tcode{move_iterator}} is a bidirectional iterator if its underlying iterator is a bidirectional iterator.}
\end{itemdescr}

\begin{itemdecl}
@\addedConcepts{template<RandomAccessIterator Iter>}@
  @\addedConcepts{concept_map RandomAccessIterator<move_iterator<Iter> > \{ \}}@
\end{itemdecl}

\begin{itemdescr}
\pnum
\addedConcepts{\reallynote Declares that a \mbox{\tcode{move_iterator}} is a random access iterator if its underlying iterator is a random access iterator.}
\end{itemdescr}

\end{paras}

\section*{Acknowledgments}
Thanks to Daniel Kr\"ugler for many helpful comments and corrections.

\bibliographystyle{plain}
\bibliography{../local}

\end{document}